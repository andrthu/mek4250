\documentclass[11pt,a4paper]{report}
\usepackage{amsmath}
\usepackage{amssymb}

\usepackage{graphicx}

\usepackage{listings}
\usepackage{color} %red, green, blue, yellow, cyan, magenta, black, white
\definecolor{mygreen}{RGB}{28,172,0} % color values Red, Green, Blue
\definecolor{mylilas}{RGB}{170,55,241}


\usepackage{graphicx}


\begin{document}
\begin{center}

\LARGE Mek4250 - Mandatory assignment 1
\\
Andreas Thune
\\
\LARGE
16.03.2016

\end{center}
\textbf{Exercise 1}
\\
a) The $H^p$ norm of a function $u(x,y)$ of two variables on $\Omega=(0,1)^2$ is defined as follows: 
\begin{gather}
||u||_{H^p(\Omega)}^2 = \sum_{i=0}^p \sum_{j=0}^{i}\binom {i} {j} ||\frac{\partial^i u}{\partial^j x \partial^{i-j}y}||_{L^2(\Omega)}^2 
\end{gather}
In our case $u(x,y)=sin(k\pi x)cos(l\pi y)$. We easily see that the derivative of this function can be expressed with the following formula: 
\begin{gather}
\frac{\partial^i u(x,y)}{\partial^j x \partial^{i}y} = (k\pi)^j(l\pi)^if_j(k\pi x)g_j(l\pi y)
\end{gather}
were $f_j$ is the j-th derivative of $sin(x)$ and $g_i$ is the i-th derivative of $cos(y)$. Now lets look at the $L^2$ norm of (2).
\begin{gather*}
||\frac{\partial^i u(x,y)}{\partial^j x \partial^{i}y}||_{L^2(\Omega)}^2 = (k\pi)^{2j}(l\pi)^{2i} \int\int_{\Omega} f_j(k\pi x)^2g_i(l\pi y)^2 dxdy \\
= (k\pi)^{2j}(l\pi)^{2i}\int_0^1f_j(k\pi x)^2dx\int_0^1g_i(l\pi y)^2dy
\end{gather*}
Since $f_j^2$ and $g_i^2$ are:
\begin{displaymath}
   f_j(x)^2 = \left\{
     \begin{array}{lr}
       sin^2(x) &  \text{j even} \\
       cos^2(x) &  \text{j odd}
     \end{array}
   \right.
\end{displaymath}
and 
\begin{displaymath}
   g_i(y)^2 = \left\{
     \begin{array}{lr}
       cos^2(y) &  \text{i even} \\
       sin^2(y) &  \text{i odd}
     \end{array}
   \right.
\end{displaymath}
and since $$\int_0^1sin^2(l\pi y)dy=\int_0^1cos^2(l\pi y)dy=\frac{1}{2} $$
we get the following expression for the square of the $L^2$ norm of a general derivative of u:
\begin{gather*}
||\frac{\partial^i u(x,y)}{\partial^j x \partial^{i}y}||_{L^2(\Omega)}^2 = \frac{1}{4}(k\pi)^{2j}(l\pi)^{2i} 
\end{gather*}
If we plug this into (1), we get
\begin{gather*}
||u||_{H^p(\Omega)}^2 = \frac{1}{4}\sum_{i=0}^p \sum_{j=0}^{i}\binom {i} {j}(k\pi)^{2j}(l\pi)^{2(i-j)}  
\end{gather*}
\\
\textbf{Exercise 2}
\\
a) Assume our solution is on the form $u(x,y)=X(x)Y(y)$. If we plug this into our equation and assume that both $X$ and $Y$ are nonzero, we get:
\begin{align*}
&-\mu(X''Y+XY'') + X'Y = 0 \iff \frac{- \mu X''+X'}{X} - \mu\frac{Y''}{Y}=0 \\
&\iff - \mu X''+X'=\lambda X \ \text{and} \ \mu Y'' = \lambda Y
\end{align*} 
Now lets look at the boundary conditions, starting withe the Dirichlet conditions:
\begin{align*}
u(0,y)=0 \iff X(0)Y(y)=0 \Rightarrow X(0)=0 
\end{align*}
Since $Y(y)=0$ would be a contradiction contradiction
\begin{align*}
u(1,y)=1 \iff X(1)Y(y)=1 \Rightarrow Y(y)=1/X(1)
\end{align*} 
This means that $Y(y)$ is a constant. This does not contradict our Neumann boundary conditions, since they say that the y-derivative is zero at $y=0$ and $y=1$. This means that our PDE is really an ODE on the form:
\begin{displaymath}
   \left\{
     \begin{array}{lr}
       -\mu X''(x)+X'(x)= 0&   \\
       X(0)= 0, \ X(1)=1 &  
     \end{array}
   \right.
\end{displaymath}
This gives us: 
\begin{align}
\mu X'(x) = X(x) + C &\iff (X(x)e^{\frac{-x}{\mu}})'=Ce^{\frac{-x}{\mu}} \\
&\iff X(x) = C' + De^{\frac{x}{\mu}}
\end{align}
Our boundary terms yields
\begin{align*}
C'&=-D\\
C'+De^{\frac{1}{\mu}}&=1
\end{align*}
The solution to this system is:
\begin{align*}
C'&=\frac{1}{1-e^{\frac{1}{\mu}}} \\
D&=-\frac{1}{1-e^{\frac{1}{\mu}}}
\end{align*}
Putting this into (8) gives us:
\begin{align*}
X(x)=\frac{1-e^{\frac{x}{\mu}}}{1-e^{\frac{1}{\mu}}}
\end{align*}
and 
\begin{align*}
u(x,y)=\frac{1-e^{\frac{x}{\mu}}}{1-e^{\frac{1}{\mu}}}
\end{align*}
\end{document}
