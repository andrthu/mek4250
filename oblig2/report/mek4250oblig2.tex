\documentclass[11pt,a4paper]{report}
\usepackage{amsmath}
\usepackage{amssymb}

\usepackage{graphicx}

\usepackage{listings}
\usepackage{color} %red, green, blue, yellow, cyan, magenta, black, white
\definecolor{mygreen}{RGB}{28,172,0} % color values Red, Green, Blue
\definecolor{mylilas}{RGB}{170,55,241}


\usepackage{graphicx}


\begin{document}
\begin{center}

\LARGE Mek4250 - Mandatory assignment 2
\\
Andreas Thune
\\
\LARGE
8.05.2016

\end{center}
\textbf{Exercise 1}
\\
7.1) Given the following weak formulation of Stokes problem:
\begin{align*}
a(u,v)+b(p,v)&=(f,v) \ \forall \ v \in H_0^1(\Omega)\\
b(q,u)&=0 \ \forall q \in L^2(\Omega)
\end{align*}
Where $a(u,v) = \int_{\Omega}\nabla u : \nabla v dx$ and $b(p,v) =\int_{\Omega} p\nabla \cdot v dx$.
\\
We then want to show the following conditions required for well-posedness:
\begin{align}
&a(u,v) \leq C_1 ||u||_{H^1(\Omega)}||v||_{H^1(\Omega)} \ \forall \ v,u \in H_0^1(\Omega)\\
&a(u,u) \geq D||u||_{H^1(\Omega)}^2 \ \forall \ u \in H_0^1(\Omega)\\
&b(q,u) \leq C_2||u||_{H^1(\Omega)}||q||_{L^2(\Omega)} \ \forall \ u \in H_0^1(\Omega), \ \forall q \in L^2(\Omega)
\end{align}
Note that $u$ and $v$ are vector functions. To show the three conditions I will use the notation $u(x)=[u^1(x),...,u^d(x)]$, for $x \in  \mathbb{R}^d$. We must also remember that the inner product $":"$ in the $a$ form is defined as 
\begin{align*}
\nabla u : \nabla v = \Sigma_{i=1}^d\Sigma_{j=1}^d u_{x_j}^i(x)v_{x_j}^i(x)
\end{align*}
\\
\textbf{Inequality (1)}
\\
\begin{align*}
a(u,v)&= \int_{\Omega}\nabla u : \nabla v dx=\int_{\Omega}\sum_{i=1}^d\sum_{j=1}^d u_{x_j}^iv_{x_j}^idx \\
&=\sum_{i=1}^d\sum_{j=1}^d (u_{x_j}^i,v_{x_j}^i)_{L^2(\Omega)} \\
&\leq \sum_{i=1}^d\sum_{j=1}^d ||u_{x_j}^i||_{L^2(\Omega)}||v_{x_j}^i||_{L^2(\Omega)} \ \text{  using Cauchy-Schwartz inequality}  \\
&\leq d \sum_{i=1}^d |u^i|_{H^1(\Omega)}|v^i|_{H^1(\Omega)} \ \text{ since $||u_{x_j}^i||_{L^2} \leq |u^i|_{H^1}$} \\
&\leq d^2 |u|_{H^1(\Omega)}|v|_{H^1(\Omega)} \ \text{        using $|u^i|_{H^1} \leq |u|_{H^1}$} \\
&\leq d^2 ||u||_{H^1(\Omega)}||v||_{H^1(\Omega)} \ \text{    since $|u|_{H^1}\leq ||u||_{H^1}$ } 
\end{align*}
\\
\textbf{Inequality (2)}
\\
Using the Poincares inequality for $H_0^1(\Omega)$, given by: 
\begin{align*}
|u|_{H^1(\Omega)}^2\geq C||u||_{L^2(\Omega)}^2
\end{align*} 
we get 
\begin{align*}
a(u,u)&= \int_{\Omega}\nabla u : \nabla u dx=\int_{\Omega}\sum_{i=1}^d\sum_{j=1}^d u_{x_j}^iu_{x_j}^idx \\
&=\sum_{i=1}^d\sum_{j=1}^d ||u_{x_j}^i||_{L^2(\Omega)}^2=\sum_{i=1}^d |u^i|_{H^1(\Omega)}^2 = |u|_{H^1(\Omega)}^2 \\
&= \frac{1}{2}(|u|_{H^1(\Omega)}^2+|u|_{H^1(\Omega)}^2) \\
&\geq \frac{1}{2}(|u|_{H^1(\Omega)}^2+C||u||_{L^2(\Omega)}^2) \\
&\geq \frac{min\{1,C\}}{2}(|u|_{H^1(\Omega)}^2+||u||_{L^2(\Omega)}^2) \\
&= D||u||_{H^1(\Omega)}^2
\end{align*}
\\
\textbf{Inequality (3)}
\\
\begin{align*}
b(q,u)&=\int_{\Omega} q\nabla \cdot u dx=\int_{\Omega}\sum_{i=1}^d qu_{x_i}^i dx \\
&\leq \sum_{i=1}^d ||q||_{L^2(\Omega)} ||u_{x_i}^i||_{L^2(\Omega)} \ \text{  using Cauchy-Schwartz inequality}  \\
&\leq ||q||_{L^2(\Omega)}\sum_{i=1}^d |u^i|_{H^1(\Omega)} \\
&\leq d||q||_{L^2(\Omega)}|u|_{H^1(\Omega)} \\
&\leq d||q||_{L^2(\Omega)}||u||_{H^1(\Omega)}
\end{align*}
\end{document}