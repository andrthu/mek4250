\documentclass[11pt,a4paper]{report}
\usepackage{amsmath}
\usepackage{amssymb}

\usepackage{graphicx}

\usepackage{listings}
\usepackage{color} %red, green, blue, yellow, cyan, magenta, black, white
\definecolor{mygreen}{RGB}{28,172,0} % color values Red, Green, Blue
\definecolor{mylilas}{RGB}{170,55,241}


\usepackage{graphicx}


\begin{document}
\begin{center}

\LARGE L-BFGS


\end{center}
\textbf{General algorithm}
\\
The BFGS and L-BFGS are quasi-Newton methods for solving unconstrained optimization problems on the form:
\begin{align*}
\min_{x \in \mathbb{R}^n} J(x)
\end{align*} 
This can looked on as solving the system of equations on the form \\$\nabla J(x) = F(x)=0$. Using the Newton method, that can be derived using Taylor series, we get the following iteration for solving the system:
\begin{align*}
x^{k+1} = x^k - F'(x^k)^{-1}F(x^k)
\end{align*}
Here $x^k \in \mathbb{R}^n$, $F:\mathbb{R}^n \rightarrow \mathbb{R}^n$ and $F':\mathbb{R}^{n\times n} \rightarrow \mathbb{R}^{n\times n}$. $F'$ is also the Hessian of $J$. The difference between Newton and BFGS is that we in BFGS approximate $F'(x)^{-1}$ with a matrix that we for each iteration update with information gained form current information. This means that we get a series of matrices $\{H^k\}$, that hopefully approximates the real hessian $H^kx^k \approx F'(x^k)^{-1}$. Without going into detail lets state the update formula for $H^k$:
\begin{align*}
H^{k+1} &= (\mathbb{I} - \frac{s_k {y_k}^T}{\rho_k})H^k(\mathbb{I} - \frac{s_k {y_k}^T}{\rho_k}) + \frac{s_k {s_k}^T}{\rho_k} \ \text{ where} \\
s_k &= x^{k+1} - x^k \\
y_k &= \nabla J(x^{k+1}) - \nabla J(x^k) \\
\rho_k &= s_k^Ty_k
\end{align*} 
We also need an initial approximation $H^0$ to the inverted hessian. This is typically set to be $H^0 = \beta\mathbb{I}$, where $\mathbb{I}$ is the identity matrix and $\beta$ is some constant. The BFGS algorithm at step $k$ then looks like:
\begin{align*}
&\text{update control by:} \ x^{k+1}= x^{k} - H^kF(x^{k}) \\
&\text{update $s_k$, $y_k$ and $\rho_k$ as described above} \\
&\text{update $H$ by:} \ H^{k+1} = (\mathbb{I} - \frac{s_k {y_k}^T}{\rho_k})H^k(\mathbb{I} - \frac{s_k {y_k}^T}{\rho_k}) + \frac{s_k {s_k}^T}{\rho_k}
\end{align*}
\textbf{L-BFGS}
\\
The difference between BFGS and L-BFGS is that one, in the L-BFGS case, base the approximation of the inverted Hessian on only the latest iterations. The size of the memory can wary. 
\\
\\
\textbf{Optimal control with ODE constraints}
\\
We want to solve an optimal control problem with ODE constraints in parallel  using the penalty approach. The problem looks as follows:
\begin{align*}
\min_{u} J(y(u),u) &= \frac{1}{2}(\int_0^T u^2 dt + (y(T)-y_T)^2) \\
y'(t) &= ay(t) +u(t)
y(0) =y_0
\end{align*} 
For the penalty approach we partition the time domain $[0,T]$ into $m+1$ intervals $[T_i,T_{i+1}]$, and solve the problem on each interval separately. To enforce continuity we add penalty terms to the $J$ functional. This leads to a new functional:
\begin{align*}
J(y,u,\lambda) = \int_0^T u^2 dt + \frac{1}{2}(y_n(T)-y_T)^2 + \frac{\mu}{2} \sum_{i=1}^n (y_{i-1}(T_i)-\lambda_i)^2
\end{align*}
Here $y_i$ is the solution of the ODE restricted to interval $i$, and $\lambda = \{ \lambda_i\}_{i=1}^n$ are the initial values of $y_i$. Using the adjoint equations $p_i$, we get the following gradient for $J$ with respect to $u$ and $\lambda$:
\begin{align*}
\langle \hat{J}'(u,\lambda), (s,l)\rangle&=\int_0^T (u+p)s \ dt +\sum_{i=1}^n(p_{i}(T_i) -p_{i-1}(T_i) )l_i \\
&= \int_0^T (u+p)s \ dt +\sum_{i=1}^n(p_{i}(T_i) -\mu(y_{i-1}(T_i)-\lambda_i) )l_i
\end{align*} 
If we discritize the time interval using $N$ points, we get the following control:
\begin{align*}
x = (u^1,...,u^N, \lambda_1, ...,\lambda_m)
\end{align*}
and the gradient:
\begin{align*}
\nabla J(x) &= (\Delta t (u^1+p^1),\Delta t (u^2+p^2),...,\Delta t (u^N+p^N),p_{1}(T_1) -p_{0}(0),..,p_{m}(T_m) -p_{m-1}(T_m)) \\
&=(\Delta t (u^1+p^1),...,\Delta t (u^N+p^N),p_{1}(T_1) -\mu(y_{0}(T_1)-\lambda_1),..,p_{m}(T_m) -\mu(y_{m-1}(T_m)-\lambda_m)) \\
&= (\Delta t\{u^j+p^j\}_{j=1}^N|p_i(T_i)) + \mu(\{0\}_{j=1}^N|\lambda_i - y_{i-1}(T_i))
\end{align*}

\end{document}