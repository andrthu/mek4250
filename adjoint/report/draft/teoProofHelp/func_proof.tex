\documentclass[11pt,a4paper]{article}
\usepackage{amsmath}
\usepackage{amssymb}
\usepackage{amsthm}
\usepackage[utf8]{inputenc}
\usepackage{graphicx}

\usepackage[utf8]{inputenc}
\usepackage[english]{babel}
 
\usepackage{cite}


\newtheorem{theorem}{Theorem}

\usepackage{listings}
\usepackage{color} %red, green, blue, yellow, cyan, magenta, black, white
\definecolor{mygreen}{RGB}{28,172,0} % color values Red, Green, Blue
\definecolor{mylilas}{RGB}{170,55,241}


\usepackage{graphicx}




\begin{document}
\section{General Problem}
Looking at an optimal control problem $$\underset{y,u}{\text{min}} \ J(y,u) \ \text{subject to} \ E(y,u)=0$$ Where $u \in U$ is the control and $y \in Y$ is the state that depends on $u$. Usually $u$ and $y$ are functions, and $U$ and $Y$ are either Hilbert or Banach spaces. I will not go into detail about these spaces, but they are mostly chosen to fit the differential equation $E$, which is an operator on $U\times Y$.
\\
\\
Differentiating $J$ is required for solving the problem. To do this we reduce $J$ to $\hat{J}(u) = J(y(u),u) $ and compute its gradient in direction $s \in U$. Will use the notation: $\langle\hat{J}'(u),s\rangle$ for the gradient.
\begin{align*}    
\langle\hat{J}'(u),s\rangle &= \langle DJ(y(u),u) ,s\rangle \\ &= \langle \frac{\partial y(u)}{\partial u}^*J_y(y(u),u),s\rangle + \langle J_u(y(u),u),s\rangle \\ &= \langle y'(u)^*J_y(u),s\rangle +\langle J_u(u),s\rangle
\end{align*}
Here $\langle\cdot,\cdot\rangle$ is the $U$ inner product. The difficult term in the expression above is $y'(u)^*$, so lets first differentiate $E(y(u),u)=0$ with respect to $u$, and try to find an expression for $y'(u)^*$: 
\begin{align*}
\frac{\partial}{\partial u}E(y(u),u)=0 &\Rightarrow E_y(y(u),u)y'(u)=-E_u(y(u),u) \\ &\Rightarrow y'(u)=-E_y(y(u),u)^{-1}E_u(y(u),u) \\ &\Rightarrow y'(u)^* = -E_u(y(u),u)^*E_y(y(u),u)^{-*}
\end{align*} 
By inserting our new expression for $y'(u)^*$ into $y'(u)^*J_y(u)$, we get:
\begin{align*}
y'(u)^*J_y(u)&=-E_u(y(u),u)^*E_y(y(u),u)^{-*}J_y(u) \\
&=-E_u(y(u),u)^*p
\end{align*}
p is here the solution of the adjoint equation 
\begin{gather*}
E_y(y(u),u)^{*}p=J_y(u)
\end{gather*}
If we can solve this equation for p, the gradient of $\hat{J}$ will be given by the following formula:  
\begin{gather}
\langle\hat{J}'(u),s\rangle=\langle -E_u(y(u),u)p,s\rangle +\langle J_u(u),s\rangle
\end{gather} 
\section{Optimal control with ODE constraints}
Lets try to derive the adjoint equation and the gradient, when we let $E(y,u)$ be the following ODE:
\begin{align*}
\left\{
     \begin{array}{lr}
       	y'(t)=\alpha y(t) +u(t), \ t \in (0,T)\\
       	   y(0)=y_0
     \end{array}
   \right.
\end{align*}
We also choose the functional to be
\begin{align*}
J(y,u) = \frac{1}{2}\int_0^Tu(t)^2dt + \frac{1}{2}(y(T)-y^T)^2
\end{align*}
\begin{theorem}
The adjoint equation of the above problem is:
\begin{align*}     
-p'(t) &= \alpha p(t) \\
p(T) &= y(T)-y^T     
\end{align*}
\end{theorem}
\begin{proof}
Before we calculate the different terms used to derive the adjoint equation, we want to fit our ODE into an expression $E$. We do this by writing up the weak formulation of the equation:
\begin{gather*}
L[y,\phi] = \int_0^T-y(t)\phi'(t)-\alpha y(t)\phi(t)dt -y_0\phi(0)+y(T)\phi(T)-\int_0^Tu(t)\phi(t)=0\\ \forall \ \phi \in C^{\infty}((0,T))
\end{gather*}
To derive the adjoint we need $E_y$ and $J_y$. For $E_y$ we define $(\cdot,\cdot)$ to be the $L^2$ inner product over $(0,T)$. This gives us:
\begin{align*}
E_y=L_y[\cdot,\phi]=(\cdot,(\frac{\partial}{-\partial t} - \alpha + \delta_T)\phi)  
\end{align*}
Lets be more thorough with $J_y$, which is the right hand side in the adjoint equation.
\begin{align*}
J_y(y(u),u) &= \frac{\partial}{\partial y}(\frac{1}{2}\int_0^Tu^2dt + \frac{1}{2}(y(T)-y^T)^2) \\ &= \frac{\partial}{\partial y} \frac{1}{2}(y(T)-y^T)^2 \\
&= \frac{\partial}{\partial y}\frac{1}{2}(\int_0^T \delta_T(y-y^T)dt)^2 \\
&= \delta_T\int_0^T \delta_T(y(t)-y^T)dt \\
&= \delta_T(y(T)-y^T)
\end{align*}
We have $E_y=(\cdot,(-\frac{\partial}{\partial t} - \alpha + \delta_T)\phi)$, but for the adjoint equation we need to find $E_y^*$.
To derive the adjoint of $E_y$, we will insert two functions $v$ and $w$ into $L_y[v,w]$, and try to change the places of $v$ and $w$.
\begin{align*}
E_y&=L_y[v,w]=\int_0^T-v(t)(w'(t)+\alpha w(t))dt + v(T)w(T) \\
&=\int_0^Tw(t)(v'(t)-\alpha v(t))dt + v(T)w(T)-v(T)w(T) +v(0)w(0) \\
&=\int_0^Tw(t)(v'(t)-\alpha v(t))dt+v(0)w(0) \\
&=L_y^*[w,v]=E_y^*
\end{align*}
If we multiply $J_y$ with a test function $\psi$ and set $L_y^*[p,\psi]=(J_y,\psi)$, we get the following equation:
\begin{align*}
\int_0^Tp(t)\psi'(t)-\alpha p(t)\psi(t)dt + p(0)\psi(0)= (y(T)-y^T)\psi(T)\ \forall \ \psi \in C^{\infty}((0,T))
\end{align*}
If we multiply then do partial integration, we get:
\begin{align*}
\int_0^T(-p'(t)-\alpha p(t))\psi(t)dt +p(T)\psi(T)= (y(T)-y^T)\psi(T)\ \forall \ \psi \in C^{\infty}((0,T))
\end{align*}
Using this we get the strong formulation:
\begin{align*}
   \left\{
     \begin{array}{lr}
       -p'(t) = \alpha p(t) \\
       p(T) = y(T)-y^T
     \end{array}
   \right.
\end{align*}
\end{proof}
With the adjoint we can find the gradient of $\hat{J}$. Lets state the result first.
\begin{theorem}
The gradient of the reduced functional $\hat{J}$ with respect to u is 
$$\hat{J}'(u)=u+p$$
\end{theorem}
\begin{proof}
Firstly we need $J_u$ and $E_u^*$:
\begin{align*}
J_u &= u \\
E_u &= L_u[\cdot,\phi] = -(\cdot,\phi)
\end{align*}
Since $R_u[\cdot,\phi]$ is symmetric, $E_u^*=E_u$, and strongly formulated, $E_u=-1$. The expression for the gradient is then simply:
\begin{align*}
\hat{J}'(u)&=-E_u^*p + J_u \\
&= p+u
\end{align*} 
the directional derivative $\langle\hat{J}'(u),s\rangle$, will therefore be:
\begin{align*}
\langle\hat{J}'(u),s\rangle =\int_0^T(p(t)+u(t))s(t)dt
\end{align*}
\end{proof}
\section*{Parallelizeing in time using the penalty method}
To find the above gradient, we must solve first the state equation forward in time and then the adjoint equation backwards in time. One way of speeding things up is to parallelize the solvers by partitioning the time interval and then solving the equation separately on each partitioned interval. If we split the interval $[0,T]$ into $m$ parts we need to solve $m$ state equations on the following form:
\begin{align*}
   \left\{
     \begin{array}{lr}
       \frac{\partial }{\partial t} y_i(t) = \alpha y_i(t) + u(t) \ \text{for $t \in [T_{i-1},T_{i}]$}\\
	y_i(T_{i-1}) = \lambda_{i-1}
     \end{array}
   \right.
\end{align*}
here $i=1,...,m$, $\lambda_0=y_0$ and $0=T_0<T_1<\cdots<T_{m}=T$. Since the equation on each interval depends on the equation in the previous interval, we need a trick, to get everything to hang together. We do this using the penalty method, which means adding a penalty to the functional. The new functional now looks like this:
\begin{align}
J(y,u,\lambda) = \int_0^T u^2 dt + \frac{1}{2}(y_m(T)-y^T)^2 + \frac{\mu}{2} \sum_{i=1}^{m-1} (y_{i}(T_i)-\lambda_i)^2 \label{penalty_func}
\end{align}
This means that the problem now is to minimize $J$ with respect to both $u$ and $\lambda$, which means that the reduced functional depends on both $u$ and $\lambda$. Since we change the functional and the equation, both the adjoint equation and the gradient changes. Lets try to derive the new adjoint equations and the new gradient. The gradient of the reduced functional now looks like the following:
\begin{align}
\langle \hat{J}'(u,\lambda), (s,l)\rangle &= \langle \frac{\partial y(u,\lambda)}{\partial(u,\lambda)}^* J_y(y(u,\lambda),u,\lambda), (s,l)\rangle + \langle J_u+J_{\lambda}, (s,l)\rangle \\
&=\langle -(E_u+E_{\lambda})p , (s,l)\rangle + \langle J_u+J_{\lambda}, (s,l)\rangle \label{pen_abs_grad}
\end{align} 
Where p is the solution of the adjoint equation $E_y^*p=J_y$, and $E$ is the collection of the interval specific state equations $E^i$. Since we now have separate state equations on each interval, the adjoint equation is also a collection of equations specific to each interval. I now state the new adjoints:
\begin{theorem}
The adjoint equation on interval $[T_{m-1},T_m]$ is:
\begin{align*}
-\frac{\partial }{\partial t}p_m &=\alpha p_m  \\
p_m(T_{m}) &= y_m(T_{m})-y_T
\end{align*}
On $[T_{i-1},T_i]$ the adjoint equation is:
\begin{align*}
-\frac{\partial }{\partial t}p_i &=p_i  \\
p_i(T_{i}) &= \mu(y_{i}(T_{i})-\lambda_{i} )
\end{align*}
\end{theorem} 
\begin{proof}
Lets begin as we did for the non-penalty approach, by writing up the weak formulation of the state equations:
\begin{gather*}
L^i[y,\phi] = \int_{T_{i-1}}^{T_{i}}-y_i(t)(\phi'(t) +\alpha \phi(t))+u(t)\phi(t)dt -\lambda_{i-1}\phi(T_{i-1})+ y_i(T_i)\phi(T_i) =0\\ \forall \ \phi \in C^{\infty}((T_{i-1},T_{i}))
\end{gather*} 
To find the adjoint equations we want to differentiate the $E^i$s and the functional $J$ with respect to $y$. To make notation easier, let $(\cdot,\cdot)_i$ be $L^2$ inner product of the interval $[T_{i-1},T_i]$. 
\begin{align*}
E_y^i=L_y^i[\cdot,\phi]=(\cdot,-(\frac{\partial}{\partial t} + \alpha - \delta_{T_i})\phi) 
\end{align*}
Lets differentiate $J$:
\begin{align*}
J_y = \delta_{T_{m}}(y_n(T_{m})-y_T) + \mu \sum_{i=1}^{m-1} \delta_{T_{i}}(y_{i}(T_i)-\lambda_i ) 
\end{align*}
Since $y$ really is a collection of functions, we can differentiate $J$ with respect to $y_i$. This gives us:
\begin{align*}
J_{y_m} &= \delta_{T_{m}}(y_n(T_{m})-y_T) \\
J_{y_i} &= \mu\delta_{T_{i}}(y_{i}(T_i)-\lambda_i ) \ i\neq m
\end{align*}
We will now find the adjoint equations, by finding the adjoint of the $E_y^i$s. This is done as above, by inserting two functions $v$, $w$ into $L_y^i[v,w]$, and then moving the derivative form $w$ to $v$.
\begin{align*}
E_y^i&=L_y^i[v,w]=\int_{T_{i-1}}^{T_i}-v(t)(w'(t)+\alpha w(t))dt + v(T_i)w(T_i) \\
&=\int_{T_{i-1}}^{T_i}w(t)(v'(t)-\alpha v(t))dt + v(T_i)w(T_i)-v(T_i)w(T_i) +v(T_{i-1})w(T_{i-1}) \\
&=\int_{T_{i-1}}^{T_i}w(t)(v'(t)-\alpha v(t))dt + v(T_{i-1})w(T_{i-1}) \\
&=(L_y^i)^*[w,v]
\end{align*}
this means that $(E_y^i)^*=(L_y^i)^*[\cdot,\psi]$. The weak form of the adjoint equations is then found, by setting setting $(L_y^i)^*[p,\psi]=(J_{y_i},\psi)_i$. This gives to cases:
\\
\\
$i=m$ case:
\begin{align*}
&\forall \ \psi \in C^{\infty}((T_{m-1},T_m)) \\
&\int_{T_m-1}^{T_m}p_m(t)\psi'(t)-\alpha p_m(t)\psi(t)dt +p_m(T_{m-1})\psi(T_{m-1})
= (y(T_m)-y^T)\psi(T_m)\ 
\end{align*}
$i\neq m$ cases:
\begin{align*}
&\forall \ \psi \in C^{\infty}((T_{i-1},T_i))\\
&\int_{T_i-1}^{T_i}p_i(t)\psi'(t)-\alpha p_i(t)\psi(t)dt +p_i(T_{i-1})\psi(T_{i-1})
= \mu(y_{i}(T_i)-\lambda_i )\psi(T_i) \ 
\end{align*}
If we want to go back to the strong formulation, we do partial integration, and get:
\\
\\
 $i=m$ case:
\begin{align*}
&\forall \ \psi \in C^{\infty}((T_{m-1},T_m)) \\
&\int_{T_m-1}^{T_m}-p_m'(t)\psi(t)-\alpha p_m(t)\psi(t)dt +p_m(T_{m})\psi(T_{m})
= (y(T_m)-y^T)\psi(T_m)\ 
\end{align*}
$i\neq m$ cases:
\begin{align*}
&\forall \ \psi \in C^{\infty}((T_{i-1},T_i))\\
&\int_{T_i-1}^{T_i}-p_i('t)\psi(t)-\alpha p_i(t)\psi(t)dt +p_i(T_{i})\psi(T_{i})
= \mu(y_{i}(T_i)-\lambda_i )\psi(T_i) \ 
\end{align*}
This gives us the ODEs we wanted.
\end{proof}
With the adjont equations we can find the gradient.
\begin{theorem}
The gradient of (\ref{penalty_func}), $\hat{J}'$, with respect to the control $(u,(\lambda_1,...,\lambda_{m-1}))$ is:
\begin{align*}
\hat{J}'(u,\lambda) = (u+p,p_{2}(T_1) -p_{1}(T_1),..., p_{m}(T_{m-1}) -p_{m}(T_{m-1}))
\end{align*} 
and the directional derivative with respect to $L^2$-norm in direction $(s,l)$ is:
\begin{align*}
\langle \hat{J}'(u,\lambda), (s,l)\rangle = \int_0^T (u+p)s \ dt +\sum_{i=1}^{m-1}(p_{i+1}(T_i) -p_{i}(T_i) )l_i
\end{align*}
\end{theorem}
\begin{proof}
If we first find $E_u^*$, $E_{\lambda}^*$, $J_u$ and $J_{\lambda}$ find the gradient by simply inserting these expression into (\ref{pen_abs_grad}). We can begin with the $E$ terms:
\begin{align*}
E_u &= L_u[\cdot,\phi] = -(\cdot,\phi) \\
E_{\lambda_{i-1}}^i &= L_{y_{i-1}}^i[\cdot,\phi] = -(\cdot,\delta_{T_{i-1}}\phi)_i
\end{align*}
Notice that both of these forms are symmetric, and we therefore don't need to do more work to find their adjoints.
\end{proof}
\begin{align*}
\langle \hat{J}'(u,\lambda), (s,l)\rangle&=\langle -(E_u+E_{\lambda})p, (s,l)\rangle + \langle J_u+J_{\lambda}, (s,l)\rangle \\
&= \langle (p+\sum_{i=1}^{m-1} \delta_{T_i}p_{i+1}) , (s,l)\rangle+ \int_0^T us \ dt - \mu \sum_{i=1}^{m-1}(y_{i}(T_i)-\lambda_i)l_i\\
&=\int_0^T (u+p)s \ dt +\sum_{i=1}^{m-1}(p_{i+1}(T_i) -\mu(y_{i}(T_i)-\lambda_i) )l_i \\
&= \int_0^T (u+p)s \ dt +\sum_{i=1}^{m-1}(p_{i+1}(T_i) -p_{i}(T_i) )l_i
\end{align*} 
\end{document}