\documentclass[11pt,a4paper]{article}
\usepackage{amsmath}
\usepackage{amssymb}
\usepackage{amsthm}
\usepackage[utf8]{inputenc}
\usepackage{graphicx}

\usepackage[utf8]{inputenc}
\usepackage[english]{babel}
 
\usepackage{cite}


\newtheorem{theorem}{Theorem}

\usepackage{listings}
\usepackage{color} %red, green, blue, yellow, cyan, magenta, black, white
\definecolor{mygreen}{RGB}{28,172,0} % color values Red, Green, Blue
\definecolor{mylilas}{RGB}{170,55,241}


\usepackage{graphicx}

\title{Parareal explanation}


\begin{document}
\maketitle
\section{Introduction}
Want to summarize the paper \cite{lions2001resolution}, and explain the parareal scheme. The parareal scheme is used to parallelize differential equations in temporal direction, by decomposing the time interval $I=[0,T]$. The equation looks like the following:
\begin{align}
\left\{
     \begin{array}{lr}
		\frac{\partial u}{\partial t} + Au = f \ 				\textit{For $t \in I$} \\
		u(0)=u_0
	\end{array}
   \right.			
\end{align} 
Decomposing the interval $I$, means dividing the interval into $N$ subintervals $\{I_n = [T^{n},T^{n+1}]\}_{n=0}^{N-1}$, with length $\Delta T = T/N$. We also define new equations equations for each interval:
\begin{align}
\left\{
     \begin{array}{lr}
		\frac{\partial u^n}{\partial t} + Au^n = f \ 				\textit{For $t \in I^n$} \\
		u^n(T^n)=\lambda^n
	\end{array}
\right.	
\end{align}
Here $\lambda^0=u_0$. If the $\lambda$s are known, we can solve the equations independently on each interval. The problem is that the $\lambda$s depend on the previous intervals, and need to be calculated by solving the equation. The parareal scheme is a way of dealing with this.
\section{The parareal scheme} 
The parareal scheme finds the $\lambda$s, by solving the equation on the entire interval using an implicit euler scheme on a very course resolution, and then using this numerical solution $Y$ at the decomposed interval boundaries $\{T^n\}_{n=1}^{N-1}$ as $\lambda$s, for the real solver $y$. We can then repeat this process, by propagating the jumps $S^n=y^{n-1}(T^n)-Y^n$ using the course solver. This creates an iteration, that looks like this:
\begin{align*}
HER \ MANGLER \ DET \ NOE
\end{align*} 
\\
\\
To illustrate how it works, I will set up the parareal scheme for a simple ODE:
\begin{align}
\left\{
     \begin{array}{lr}
		\frac{\partial y}{\partial t}(t)=-ay(t) \ 				\textit{on $[0,T]$} \\
		y(0)=y_0
	\end{array}
\right.	\label{ODE_eks}
\end{align}
If we discretize \ref{ODE_eks} using implicit euler, we get:
\begin{align}
\left\{
     \begin{array}{lr}
		\frac{Y^{n+1}-Y^{n}}{\Delta T}+aY^{n+1}=0  \\
		Y^0=y_0
	\end{array}
\right.	
\end{align}
Notice that the interval $I$, is discretized using the same time difference as the time decomposition.
\bibliography{ppaper}
\bibliographystyle{plain}
\end{document}