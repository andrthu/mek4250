\documentclass[11pt,a4paper]{article}
\usepackage{amsmath}
\usepackage{amssymb}
\usepackage{amsthm}
\usepackage[utf8]{inputenc}
\usepackage{graphicx}

\usepackage[utf8]{inputenc}
\usepackage[english]{babel}
 
\usepackage{cite}


\newtheorem{theorem}{Theorem}

\usepackage{listings}
\usepackage{color} %red, green, blue, yellow, cyan, magenta, black, white
\definecolor{mygreen}{RGB}{28,172,0} % color values Red, Green, Blue
\definecolor{mylilas}{RGB}{170,55,241}


\usepackage{graphicx}

\title{Parareal explanation}


\begin{document}
\maketitle
\section{Introduction}
Want to summarize the paper \cite{lions2001resolution}, and explain the parareal scheme. The parareal scheme is used to parallelize differential equations in temporal direction, by decomposing the time interval $I=[0,T]$. The equation looks like the following:
\begin{align}
\left\{
     \begin{array}{lr}
		\frac{\partial u}{\partial t} + Au = f \ 				\textit{For $t \in I$} \\
		u(0)=u_0
	\end{array}
   \right.			
\end{align} 
Decomposing the interval $I$, means dividing the interval into $N$ subintervals $\{I_n = [T^{n},T^{n+1}]\}_{n=0}^{N-1}$, with length $\Delta T = T/N$. We also define new equations equations for each interval:
\begin{align}
\left\{
     \begin{array}{lr}
		\frac{\partial u^n}{\partial t} + Au^n = f \ 				\textit{For $t \in I^n$} \\
		u^n(T^n)=\lambda^n
	\end{array}
   \right.			
\end{align}
\bibliography{ppaper}
\bibliographystyle{plain}
\end{document}