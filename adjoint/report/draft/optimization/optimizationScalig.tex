\documentclass[11pt,a4paper]{article}
\usepackage{amsmath}
\usepackage{amssymb}
\usepackage{amsthm}
\usepackage[utf8]{inputenc}
\usepackage{graphicx}

\usepackage[utf8]{inputenc}
\usepackage[english]{babel}
\usepackage{booktabs}
\usepackage{cite}


\newtheorem{theorem}{Theorem}

\usepackage{listings}
\usepackage{color} %red, green, blue, yellow, cyan, magenta, black, white
\definecolor{mygreen}{RGB}{28,172,0} % color values Red, Green, Blue
\definecolor{mylilas}{RGB}{170,55,241}


\usepackage{graphicx}

\title{Optimization}


\begin{document}
\section{Steepest descent}
The steepest descent method is a very simple line search method that is defined by the following iteration:
\begin{align*}
x^{k+1} = x^k - \alpha\nabla J(x^k)
\end{align*}
If we want to use this method to solve a timedecomposed problem using penalty terms, we would get the following:
\begin{align*}
(v^{k+1},\lambda^{k+1})= (v^k,\lambda^k)-\alpha\nabla J(v^k,\lambda^k)
\end{align*}
One problem with the steepest descent method, is that it is sensitive to bad scaling, and in many cases the size of the gradient with respect to $v$ and $\lambda$ may have a difference of several orders of magnitude. This would result in extremely slow convergence. One way of handling this is to split up the update:
\begin{align*}
v^{k+1} &= v^k - \alpha\nabla J(v^k)\\
\lambda^{k+1} &= \lambda^k - \alpha\nabla J(\lambda^k)
\end{align*}
To illustrate that we have scaling issues, lets look at the DE constrained optimal control problem that we have looked at previously:
\begin{align}
	\left\{
     \begin{array}{lr}
		\frac{\partial y}{\partial t}(t)+Ay=Bv  \ \textit{For $t \in [0,T]$}\\
		y(0)=y^0
	\end{array}
	\right. \label{OC_PDE}
\end{align}
The functional they use is on the following form:
\begin{align}
J(v) = \frac{1}{2}\int_0^T ||v(t)||^2 dt + \frac{\alpha}{2}||y(T)-y^T||^2 \label{OC_func}
\end{align}
This problem yields the following gradient:
\begin{align}
\langle\hat{J}'(v,\lambda),(s,l) \rangle = \int_0^T (B^*p+v)sdt + \sum_{n=1}^{N-1}(p_n(T_n) - p_{n-1}(T_n))l_n \label{penalty grad}
\end{align}
In a numerical setting, where we have discretized the equation, we will get the following gradient:
\begin{align*}
\nabla J(\bar{v},\Lambda) = (\Delta t (v_0-B^*p_0),...,\Delta t (v_{M}-B^*p_M),\{p_n(T_n) - p_{n-1}(T_n)\})
\end{align*}
One sees that the adjoint appears in both parts of the gradient, but in the $v$ part there is a factor of $\Delta t$ multiplied with it. It therefore makes sense, that the there is a factor of $\Delta t$ order of magnitude difference between the size of $v$ and $\Lambda$ part of the gradient. This means that for small $\Delta t$ the steepest descent method will converge very slowly.
\section{Scaling}
An alternative way of handling an unbalanced gradient is to rescale the problem. Lets explain with a simple example, where we try to minimize a function with only two variables $J(x,y)$. We then pick some initial point $(m_1,m_2)$, and notice that the gradient $\nabla J(m_1,m_2)$ is very unbalanced. Ideally, we would want $J_x(m_1)=J_y(m_2)$, and to achieve this we introduce a new functional $\hat{J}(x,\xi)=J(x,\gamma\xi)$, with 
\begin{align*}
\gamma = \frac{\frac{\partial J(m_1,m_2)}{\partial x}}{\frac{\partial J(m_1,m_2)}{\partial y}}
\end{align*} 
Now choose a new initial point $(m_1,\frac{m_2}{\gamma})$, and we see that 
\begin{align*}
\frac{\partial \hat{J}(m_1,\frac{m_2}{\gamma})}{\partial x}=\frac{\partial\hat{J}(m_1,\frac{m_2}{\gamma})}{\partial y}
\end{align*}
because
\begin{align*}
\frac{\partial \hat{J}(m_1,\frac{m_2}{\gamma})}{\partial x} = \frac{\partial J(m_1,m_2)}{\partial x}
\end{align*}
and
\begin{align*}
\frac{\partial\hat{J}(m_1,\frac{m_2}{\gamma})}{\partial y} & = \frac{\partial J(m_1,m_2)}{\partial y}\gamma \\
&=\frac{\partial J(m_1,m_2)}{\partial x}
\end{align*}
\section{Scaling for Steepest descent}
To illustrate how scaling of the time-decomposed problem affects the performance of the steepest descent method, lets look at a simple ODE example of problem (\ref{OC_PDE}-\ref{OC_func}), with parameters $(A,B,T,\alpha,y_0,y_T,\Delta t)=(0.9,1,1,0.5,1.2,5,\frac{1}{500})$. Solving this problem using penalty term $\mu=1$ and different time-decompositions yielded the following:
\\
\begin{tabular}{llllrr}
\toprule
{}decompositions &     $\gamma$ & scaled error & scaled iterations &  error &  iterations \\
\midrule
0  &        -- &           -- &                -- &        0.000000 &                   34 \\
2  &  0.027027 &     0.224357 &                21 &        0.275448 &                  601 \\
4  &      0.04 &     0.425371 &                32 &        0.553890 &                  601 \\
8  &      0.04 &     0.551151 &                79 &        0.609155 &                  601 \\
16 &      0.04 &     0.620472 &               219 &        0.659912 &                  460 \\
32 &      0.04 &     0.657101 &               555 &        0.674689 &                  601 \\
\bottomrule
\end{tabular}
From the table above it is clear that scaling has a huge impact on the performance of the steepest descent algorithm for the time-decomposed problem. Another thing of note is the value of the scaling factor $\gamma$. Since the functional have more than two arguments, we need to choose a way to determine $\gamma$. For the example above, I have used the following:
\begin{align*}
\gamma = 20\frac{||\nabla J(v,)||_{\infty}}{||\nabla J(\Lambda)||_{\infty}}
\end{align*} 
One can choose a different factor than twenty to multiply with the fraction, what factor to choose is not obvious. Below I have tested the same problem for different time-decompositions $m$, using different factors to calculate $\gamma$. The resulyts were the following:
\\
\begin{tabular}{lllll}
\toprule
{} & m=16 (iter,gamma) &         m=2 (iter,gamma) & m=4 (iter,gamma) & m=8 (iter,gamma) \\
\midrule
1   &      (501, 0.002) &  (501, 0.00135135135135) &     (501, 0.002) &     (501, 0.002) \\
5   &       (501, 0.01) &   (99, 0.00675675675676) &       (83, 0.01) &      (241, 0.01) \\
10  &       (196, 0.02) &    (30, 0.0135135135135) &       (32, 0.02) &       (80, 0.02) \\
15  &       (188, 0.03) &    (32, 0.0202702702703) &       (40, 0.03) &       (67, 0.03) \\
20  &       (219, 0.04) &     (21, 0.027027027027) &       (32, 0.04) &       (79, 0.04) \\
30  &       (174, 0.06) &    (42, 0.0405405405405) &       (45, 0.06) &       (73, 0.06) \\
50  &        (155, 0.1) &    (28, 0.0675675675676) &        (32, 0.1) &        (65, 0.1) \\
100 &        (155, 0.2) &     (29, 0.135135135135) &        (47, 0.2) &        (84, 0.2) \\
\bottomrule
\end{tabular}
It could also be interesting to look at what happens when we decrease $\Delta t$. I have therefore looked at at the performance of different $\gamma$ values for different mesh sizes, for the problem above decomposed into for time intervals. This gave me the following:
\\
\begin{tabular}{llllll}
\toprule
{} & N=100(itr,$\gamma$,err) & N=1000(itr,$\gamma$,err) & N=200(itr,$\gamma$,err) & N=2000(itr,$\gamma$,err) & N=800(itr,$\gamma$,err) \\
\midrule
-- &               (41, --) &                (13, --) &               (16, --) &                (20, --) &               (11, --) \\
1        &       (101, 0.01, 0.6) &       (101, 0.001, 1.4) &      (101, 0.005, 1.2) &      (101, 0.0005, 1.4) &    (101, 0.00125, 1.4) \\
10       &         (26, 0.1, 0.4) &         (71, 0.01, 0.4) &        (26, 0.05, 0.4) &        (65, 0.005, 0.4) &      (63, 0.0125, 0.4) \\
15       &        (48, 0.15, 0.4) &        (43, 0.015, 0.4) &       (39, 0.075, 0.4) &       (41, 0.0075, 0.4) &     (31, 0.01875, 0.4) \\
20       &         (37, 0.2, 0.4) &         (26, 0.02, 0.4) &         (25, 0.1, 0.4) &         (26, 0.01, 0.4) &       (22, 0.025, 0.4) \\
50       &         (83, 0.5, 0.4) &         (24, 0.05, 0.4) &        (64, 0.25, 0.4) &        (18, 0.025, 0.4) &      (24, 0.0625, 0.4) \\
100      &        (101, 1.0, 0.6) &          (29, 0.1, 0.4) &        (101, 0.5, 0.5) &         (26, 0.05, 0.4) &       (39, 0.125, 0.4) \\
\bottomrule
\end{tabular}
\section{L-BFGS}

\end{document}