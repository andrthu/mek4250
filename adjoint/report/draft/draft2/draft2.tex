\documentclass[11pt,a4paper]{article}
\usepackage{amsmath}
\usepackage{amssymb}
\usepackage{amsthm}
\usepackage[utf8]{inputenc}
\usepackage{graphicx}

\usepackage[utf8]{inputenc}
\usepackage[english]{babel}
 
\usepackage{cite}


\newtheorem{theorem}{Theorem}

\usepackage{listings}
\usepackage{color} %red, green, blue, yellow, cyan, magenta, black, white
\definecolor{mygreen}{RGB}{28,172,0} % color values Red, Green, Blue
\definecolor{mylilas}{RGB}{170,55,241}


\usepackage{graphicx}
\date{\today}
\title{Optimal control with DE constraints}



\begin{document}
\maketitle
\tableofcontents{}

\section{Introduction}
\subsection{Optimal control with DE constraints}
\subsubsection{General optimal control problem}
In this thesis we are only looking at optimal control problems with time-dependent differential equation constraints. This problem is only a part of the more general control problem, which can be formulated as:
\begin{align}
\underset{y\in Y,v\in V}{\text{min}} \ &J(y,v) \\
\textit{Subject to:} \ &E(y,v)=0
\end{align}
Here $J: Y\times V\rightarrow\Re$ is the objective function that we want to minimize, while $E:Y\times V \rightarrow Z$, is an operator such that $\forall v \in V$, $\exists! y(V)\in Y$ that satisfies the state equation:
\begin{align*}
E(y(v),v)=0
\end{align*}
Replacing $y$ with $y(v)$ allows us to define the reduced problem:
\begin{align}
\underset{v\in V}{\text{min}} \ \hat J(y(v),v) \ \textit{Subject to:} \ E(y(v),v)=0 \label{reduced problem}
\end{align}
To find a solution to (\ref{reduced problem}), we need to be able to differentiate the objective function $\hat{J}$ with respect to the control $v$. There are different ways of doing that, but I will focus on the so called adjoint approach, which is the most computational effective way of calculating the gradient of $\hat{J}$.
\subsubsection{The adjoint equation and the gradient}
To find the gradient $\hat{J}'(u)$, lets start by differentiating $\hat J$:
\begin{align*}
\hat{J}'(u) = DJ(y(u),u) = y'(u)^*J_y(y(u),u) + J_u(y(u),u)
\end{align*}
The problematic term in the above expression, is $y'(u)^*$. To calcualte this we need to differentiate the state equation:
\begin{align*}
DE(y(u),u)=0 &\Rightarrow E_y(y(u),u)y'(u)=-E_u(y(u),u) \\ &\Rightarrow y'(u)=-E_y(y(u),u)^{-1}E_u(y(u),u) \\ &\Rightarrow y'(u)^* = -E_u(y(u),u)^*E_y(y(u),u)^{-*}
\end{align*}
Instead of inserting $y'(u)^* = -E_u(y(u),u)^*E_y(y(u),u)^{-*}$ into our gradient expression, we first define the adjoint equation as:
\begin{align}
E_y(y(u),u)^{*}p=J_y(u) \label{general adjoint}
\end{align}
This now allows us to write up the gradient as follows:
\begin{align}
\hat{J}'(u)&= y'(u)^*J_y(y(u),u) + J_u(y(u),u)\\
&=-E_u(y(u),u)^*E_y(y(u),u)^{-*}J_y(y(u),u) + J_u(y(u),u) \\
&= -E_u(y(u),u)^*p +J_u(y(u),u) \label{gradient}
\end{align}

\end{document}