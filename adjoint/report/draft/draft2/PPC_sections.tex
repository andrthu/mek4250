\documentclass[11pt,a4paper]{article}
\usepackage{amsmath,amsfonts,amssymb,amsthm, gensymb}
\usepackage{amssymb}
\usepackage{amsthm}
\usepackage[utf8]{inputenc}
\usepackage{graphicx}
\usepackage{color, array, threeparttable}

\usepackage[utf8]{inputenc}
\usepackage[english]{babel}
 
\usepackage{cite}

\usepackage[ruled]{algorithm2e}


\usepackage{listings}
\usepackage{color} %red, green, blue, yellow, cyan, magenta, black, white
\definecolor{mygreen}{RGB}{28,172,0} % color values Red, Green, Blue
\definecolor{mylilas}{RGB}{170,55,241}
\DeclareMathAlphabet{\mathbbold}{U}{bbold}{m}{n}    

\newtheorem{theorem}{Theorem}
\newtheorem{definition}{Definition}
\newtheorem{proposition}{Proposition}


\usepackage{graphicx}

%\cite{maday2002parareal}
% 4.3 \ref{algebraic_sec}
\begin{document}
\section{definition}
\begin{definition}[Fine and coarse propagator]
Let $f(u(t),t)=0$ be a time dependent differential equation. Given $\Delta T$ and an initial condition $\omega$, let $u_f$ and $u_c$ be a fine and a coarse numerical solution of the initial value problem:
\begin{align}
 \left\{
     \begin{array}{lr}
		f(u(t),t)=0 \ \quad \textrm{For $t \in (0,\Delta T)$} \\
		u(0)=\omega
	\end{array}
	\right.	
\end{align}
We then define the fine propagator as $\bold F_{\Delta T}(\omega) = u_f(\Delta T)$ and the coarse propagator as $\bold G_{\Delta T}(\omega) = u_c(\Delta T)$.
\end{definition}
\section{PPC}
\subsection{Virtual problem} \label{vir_sec}
The Parareal-based preconditioner only affects the part of the gradient connected to the virtual control $\Lambda$. To motivate and derive $Q$, we therefore consider an optimal control problem where the real control $v$ is removed, and the objective function only depends on $\Lambda$. We have already presented this problem in section \ref{algebraic_sec}, but we restate it here for future reference. However, before we do this let us first properly define the fine and coarse propagators.
\begin{definition}[Fine and coarse propagator] \label{prop_def}
Let $f(y(t),t)=0$ be a time dependent differential equation. Given $\Delta T=\frac{T}{N}$ and an initial condition $\omega$, let $y_f$ and $y_c$ be a fine and a coarse numerical solution of the initial value problem:
\begin{align}
 \left\{
     \begin{array}{lr}
		f(y(t),t)=0 \ \quad \textrm{For $t \in (0,\Delta T)$} \\
		y(0)=\omega
	\end{array}
	\right.	
\end{align}
We then define the fine propagator as $\bold F_{\Delta T}(\omega) = y_f(\Delta T)$ and the coarse propagator as $\bold G_{\Delta T}(\omega) = y_c(\Delta T)$. We also define the lower triangonal matrices $M,\bar M\in\mathbb{R}^{N-1\times N-1}$ as: 
\begin{align*}
M= \left[ \begin{array}{cccc}
   \mathbbold{1} & 0 & \cdots & 0 \\  
   -\bold{F}_{\Delta T} & \mathbbold{1} & 0 & \cdots \\ 
   0 &-\bold{F}_{\Delta T} & \mathbbold{1}  & \cdots \\
   0 &\cdots &-\bold{F}_{\Delta T} & \mathbbold{1}  \\
   \end{array}  \right],
\bar M= \left[ \begin{array}{cccc}
   \mathbbold{1} & 0 & \cdots & 0 \\  
   -\bold{G}_{\Delta T} & \mathbbold{1} & 0 & \cdots \\ 
   0 &-\bold{G}_{\Delta T} & \mathbbold{1}  & \cdots \\
   0 &\cdots &-\bold{G}_{\Delta T} & \mathbbold{1}   \\
   \end{array}  \right].
\end{align*}
\end{definition}
We then use the fine propagator $\bold F_{\Delta T}(\omega)$ to define the virtual problem.
\begin{definition}[Virtual problem]
Given a fine propagator $\bold F_{\Delta T}$, that solves a time dependent differential equation $f(y(t),t)=0$, an initial condition $\lambda_0=y_0$ and the control variable $\Lambda=(\lambda_1,...,\lambda_ {N-1})$, the virtual control problem is defined as follows:
\begin{align}
&\min_{\Lambda}\bold{J}(\Lambda,y) = \sum_{i=1}^{N-1} (y_{i-1}(T_{i})-\lambda_{i})^2 \label{virtual_func} \\
&\textrm{Subject to } \ y_{i-1}(T_{i}) = \bold F_{\Delta T}(\lambda_{i-1}) \ i=1,...,N-1 \label{virtual}
\end{align}
\end{definition}
In chapter \ref{parareal_chap} we explained how the virtual problem could be solved by setting $\lambda_i= \bold F_{\Delta T}(\lambda_{i-1})$, which is the same as solving $\bold J(\Lambda,y)=0$. This equation could be written up on matrix form as:
\begin{align}
M \ \Lambda = H. \label{Parareal_equation}
\end{align}
The $H$ on right hand side of the above equation is the propagator applied to the initial condition:
\begin{align*}
H = \left[ \begin{array}{c}
   \bold F_{\Delta T}( y_0) \\
   0 \\
   \cdots \\
   0 \\
   \end{array}  \right].
\end{align*}
In section \ref{algebraic_sec} we explained how the Parareal algorithm could be reformulated as a preconditioned fix point iteration solving equation (\ref{Parareal_equation}), expressed as follows:
\begin{align}
\Lambda^{k+1} = \Lambda^k + \bar{M}^{-1}(H-M\Lambda^k)\label{par_mat_sys}
\end{align}
Where $\bar{M}$ is the coarse version of the matrix $M$ stated in definition \ref{prop_def}. When we are solving the original optimal control problem we do not try to find a triple $(v,\Lambda,y)$ that solves $J_{\mu}(v,\Lambda,y)=0$. Instead we try to solve $\hat J_{\mu}'(v,\Lambda)=0$. To find the Parareal-based preconditioner, we therefore try to find a similar expression to (\ref{Parareal_equation}) for $\bold{\hat{J}}'(\Lambda)=0$. To be able to find this expression, we first need to define the coarse and fine adjoint propagators.
\begin{definition}[Fine and coarse adjoint propagator] \label{adjoint_prop_def}
Let $f(y(t),t)=0$ be a time dependent differential equation. Given $\Delta T$ a state $y(t)$ and an initial condition $\omega$, let $p_f$ and $p_c$ be a fine and a coarse numerical solution of the initial value problem:
\begin{align}
 \left\{
     \begin{array}{lr}
		f'(y(t),t)^*p(t)=0 \ \quad \textrm{For $t \in (0,\Delta T)$} \\
		p(\Delta T)=\omega
	\end{array}
	\right.	
\end{align}
We then define the fine adjoint propagator as $\bold F_{\Delta T}^*(\omega) = p_f(0)$ and the coarse adjoint propagator as $\bold G_{\Delta T}^*(\omega) = p_c(0)$. We also define adjoint versions of the matrices $M$ and $\bar M$ as: 
\begin{align*}
M^*= \left[ \begin{array}{cccc}
   \mathbbold{1} & -\bold{F}_{\Delta T}^* & 0 & 0 \\  
   0 & \mathbbold{1} & -\bold{F}_{\Delta T}^* & \cdots \\ 
   \cdots &0 &  \mathbbold{1} & -\bold{F}_{\Delta T}^* \\
   0 &\cdots &\cdots &  \mathbbold{1}  \\
   \end{array}  \right],
\bar M^*= \left[ \begin{array}{cccc}
   \mathbbold{1} & -\bold{G}_{\Delta T}^* & 0 & 0 \\  
   0 & \mathbbold{1} & -\bold{G}_{\Delta T}^* & \cdots \\ 
   \cdots &0 &  \mathbbold{1} & -\bold{G}_{\Delta T}^* \\
   0 &\cdots &\cdots &  \mathbbold{1}  \\
   \end{array}  \right].
\end{align*}
\end{definition} 
Using the matrices from definition \ref{adjoint_prop_def} we can write up the following proposition concerning the gradient of the reduced objective function of the virtual problem.
\begin{proposition} \label{vir_grad_prop}
The reduced objective function of the virtual problem (\ref{virtual_func}-\ref{virtual}) is:
\begin{align}
\bold{\hat J}(\Lambda) = \sum_{i=1}^{N-1} (\bold F_{\Delta T}(\lambda_{i-1})-\lambda_{i})^2.\label{reduced_viritual}
\end{align}
Solving $\bold{\hat J}'(\Lambda)=0$ is equivalent to resolving the system:
\begin{align}
M^* \ M \ \Lambda \ = \ M^* \ H. \label{vir_grad_sys}
\end{align}
A preconditioned fix point iteration for equation (\ref{vir_grad_sys}) inspired by the Parareal formulation (\ref{par_mat_sys}) is therefore:
\begin{align}
\Lambda^{k+1} = \Lambda^k + \bar{M}^{-1}\bar M^{-*}(M^*H-M^*M\Lambda^k). \label{grad_fix_iter}
\end{align}
\end{proposition}
\begin{proof}
Luckily for us we have already derived the gradient of $\bold{\hat J}$ in (\ref{penalty grad}). There we stated the gradient for the penalized version of the example problem (\ref{exs_J}-\ref{exs_E}). If we ignore the part of this gradient related to the real control $v$, we find get the following expression for $\bold{\hat J}'$:
\begin{align*}
\hat{\bold J}'(\Lambda) = \{p_{i+1}(T_i)-p_{i}(T_i)\}_{i=1}^{N-1}.
\end{align*}
Here $p_i$ refers to the decomposed adjoint equation on interval $[T_{i-1},T_{i}]$. We now want to show that setting $p_{i+1}(T_i)-p_{i}(T_i)=0$ for $i=1,...,N-1$ is equivalent to equation \ref{vir_grad_sys}. To do this we will simply write out the expression $M^*(M\Lambda-H)$ and show that it equals $\hat{\bold J}'(\Lambda)$. We start with $M\Lambda-H$.
\begin{align*}
M \ \Lambda - H  = \left( \begin{array}{c}
	\lambda_1-\bold{F}_{\Delta T}(\lambda_0)\\
	\lambda_2-\bold{F}_{\Delta T}(\lambda_1) \\
	\cdots \\
	\lambda_{N-1}-\bold{F}_{\Delta T}(\lambda_{N-1}) 
	\end{array} \right).
\end{align*}
Notice that $\bold{F}_{\Delta T}(\lambda_{i-1})-\lambda_i$ is the initial condition of $i$-th adjoint equation, i.e. $p_i(T_i)=\bold{F}_{\Delta T}(\lambda_{i-1})-\lambda_i$. By exploiting this, and multiplying $M\Lambda-H$ with $M^*$ we get:
\begin{align}
M^* (M \ \Lambda-H)&=
	\left( \begin{array}{c}
	 \bold{F}_{\Delta T}^*(p_2( T_2))-p_1(T_1)\\
	\bold{F}_{\Delta T}^*(p_3( T_3))-p_2(T_2)\\
	\cdots \\
	-p_{N-1}(T_{N-1})
	\end{array} \right)
	\\
	&=\left( \begin{array}{c}
	p_2(T_1)-p_1(T_1)\\
	p_3(T_2)-p_2(T_2)\\
	\cdots \\
	p_{N-1}(T_{N-2})-p_{N-2}(T_{N-2}) \\
	-p_{N-1}(T_{N-1})
	\end{array} \right).
\end{align}
The last step is done by using $p_i(T_{i-1})=-F_{\Delta T}^*(-p_i(T_i))$, and this is possible since the adjoint equation is always linear. We see that the $i$-th component of $M^* (M \Lambda-H)$ is equal to $p_{i+1}(T_i)-p_{i}(T_i)$ for $i\neq N-1$. The last component of $M^* (M \Lambda-H)$ is $-p_{N-1}(T_{N-1})$, and we are therefore missing $p_N(T_{N-1})$. This is however unproblematic since in context of the the virtual problem $p_N(T_{N-1})=0$. This shows us that $\hat{\bold J}'(\Lambda)= M^* (M \Lambda-H)$, which means that $\hat{\bold J}'(\Lambda)=0 \iff M^*M\Lambda =M^*H$. Since $\bar M$ and $\bar M^*$ approximates $M$ and $M^*$, $\bar{M}^{-1}\bar M^{-*}$ would be a natural preconditioner for a fix point iteration solving $M^*M\Lambda =M^*H$. 
\end{proof}
Proposition \ref{vir_grad_prop} motivates $Q_{\Lambda}=\bar{M}^{-1}\bar M^{-*}$ as a preconditioner for solvers of decomposed and penalized optimal control problems, and this is actually the Parareal-based preconditioner proposed in \cite{maday2002parareal}. Inserting $Q_{\Lambda}$ into $Q$ yields the following:
\begin{align}
Q = \left[ \begin{array}{cc}
	\mathbbold{1} & 0 \\
	0 &  \bar{M}^{-1}\bar{M}^{-*}\\
	\end{array} \right]. \label{Q_PC}
\end{align}  
In \cite{maday2002parareal} $Q$ is proposed as a preconditioner for a steepest descent method. We do however not know if $Q$ is positive definite, or if it is in any shape or form related to the Hessian of the objective function. We will investigate these questions further by reformulating the reduced objective function (\ref{reduced_viritual}) for the virtual problem to a least squares problem.
\subsection{Virtual lest squares problem}
Looking at the equation $M^*M\Lambda =M^*H$ we recognize the normal equation, which is connected to linear least squares problems. We therefore suspect that the virtual problem can be reformulated as a least squares problem. It turns out that this is indeed the case. We write up the new formulation in definition \ref{VLSPD}.
\begin{definition}[Virtual least squares problem] \label{VLSPD}
Given a propagator $\bold F_{\Delta T}$ as defined in definition \ref{prop_def} and an initial condition $\lambda_0=y_0$ for the state equation, the least squares formulation of the virtual optimal control problem (\ref{virtual_func}-\ref{virtual}) reads as follows:
\begin{align}
\min_{\Lambda\in\mathbb{R}^{N-1}}\hat{\bold J}(\Lambda) = x(\Lambda)^Tx(\Lambda), \label{non_lin_LS}
\end{align}
where the vector function $x:\mathbb{R}^{N-1}\rightarrow \mathbb{R}^{N-1}$ is:
\begin{align}
x(\Lambda)= \left( \begin{array}{c}  
   \lambda_1 - \bold F_{\Delta T}(\lambda_0) \\ 
   \lambda_2 - \bold F_{\Delta T}(\lambda_1) \\
   \cdots  \\
   \lambda_{N-1} -\bold F_{\Delta T}(\lambda_{N-1}) \\
   \end{array}  \right).
\end{align}
\end{definition}
We are now interested in finding the Hessian of $\hat{\bold J}(\Lambda)$, which we hope to relate to the Parareal-based preconditioner. 
\begin{proposition}\label{NonLin_prop}
The Hessian of function (\ref{non_lin_LS}) is
\begin{align*}
\nabla^2 \hat{\bold J}(\Lambda) &= 2\nabla x^T\nabla x + 2\sum_{i=1}^{N-1} \nabla^2 x_i(\Lambda) x_i(\Lambda)\\
&=2M(\Lambda)^TM(\Lambda) + 2D(\Lambda)
\end{align*}
Here $D(\Lambda)$ is a diagonal matrix with diagonal entries 
\begin{align*}
D_i=-\bold{F}_{\Delta T}''(\lambda_i)(\lambda_{i+1}-\bold F_{\Delta T}(\lambda_i)) \quad i=1,...,N-1,
\end{align*}
while $M(\Lambda)$ is the linearised forward model:
\begin{align*}
M(\Lambda) &= \left[ \begin{array}{cccc}
   \mathbbold{1} & 0 & \cdots & 0 \\  
   -\bold{F}_{\Delta T}'(\lambda_{1}) & \mathbbold{1} & 0 & \cdots \\ 
   0 &-\bold{F}_{\Delta T}'(\lambda_{2}) & \mathbbold{1}  & \cdots \\
   0 &\cdots &-\bold{F}_{\Delta T}'(\lambda_{N-1}) & \mathbbold{1}  \\
   \end{array}  \right]
\end{align*}	
\end{proposition}
\begin{proof}
We start by differentiating $\hat{\bold J}$:
\begin{align*}
\nabla \hat{\bold J}(\Lambda) &= 2 \nabla x(\Lambda)^T x(\Lambda)\\
&=2\sum_{i=1}^{N-1} \nabla x_i(\Lambda) x_i(\Lambda)
\end{align*}
If we now differentiate $\nabla \hat{\bold J}$, we get:
\begin{align*}
\nabla^2 \hat{\bold J}(\Lambda) &= 2\nabla x^T\nabla x + 2\sum_{i=1}^{N-1} \nabla^2 x_i(\Lambda) x_i(\Lambda)
\end{align*}
We see that $\nabla x(\Lambda)=M(\Lambda)$, by looking at $\frac{\partial x_i}{\partial \lambda_j}$
\begin{align*}
\frac{\partial x_i}{\partial \lambda_j} = \left\{
     \begin{array}{lr}
		1 \quad\quad\quad\quad\quad i=j\\
		-\bold F_{\Delta T}'(\lambda_{j}) \quad i>1 \wedge j=i-1 \\
		0 \quad\quad\quad\quad\quad i\neq j \vee j\neq i-1
	\end{array}
   \right.	
\end{align*}
We can similarly find $\nabla^2 x_i$ by differentiating $x$ twice:
\begin{align*}
\frac{\partial^2 x_i}{\partial \lambda_j\partial\lambda_k} = \left\{
     \begin{array}{lr}
		-\bold F_{\Delta T}''(\lambda_{j}) \quad i>1 \wedge j=k=i-1 \\
		0 \quad\textrm{in all other cases}
	\end{array}
   \right.	
\end{align*}
Now summing up the terms $\nabla^2 x_i(\Lambda)x_i(\Lambda)$ would yield the diagonal matrix $D(\Lambda)$ described in proposition \ref{NonLin_prop}.
\end{proof}
The first term of $\nabla^2 \hat{\bold J}(\Lambda)=2M(\Lambda)^TM(\Lambda) + 2D(\Lambda)$ resembles $M^*M$ from the previous section, while the second term $2D(\Lambda)$ is new. $D(\Lambda)$ is a diagonal matrix where the diagonal entries consists of products between the second derivative of $\bold F_ {\Delta T}$ and the residuals $\lambda_{i+1}-\bold F_{\Delta T}(\lambda_i)$. If the governing equation of the propagator $\bold F_ {\Delta T}$ is linear, $\bold F_{\Delta T}''(\lambda_i)=0$. This would again mean that $D(\Lambda)=0$ and that $\nabla^2 \hat{\bold J}(\Lambda)=2M(\Lambda)^TM(\Lambda)$. We will therefore split our discussion of the Hessian of $\hat{\bold J}$ into two cases. In the first we assume the sate equation is linear, while in the second case we discuss non-linear state equations.
\subsubsection{Linear state equations}
Assuming that the state equation is linear means that $\nabla^2 \hat{\bold J}(\Lambda)=2M(\Lambda)^TM(\Lambda)$. Differentiating the propagator $\bold F_{\Delta T}$ is the same as linearising its governing equation. When the governing equation is it self linear, linearising it does not change the equation. Therefore $\bold F_{\Delta T}'(\lambda_i)\lambda_i = \bold F_{\Delta T}(\lambda_i)$. This means that the $M$ matrix from section \ref{vir_sec} is equal to $M(\Lambda)$. The same is true for $M^*$ and $M(\Lambda)^T$. Since  $\nabla^2 \hat{\bold J}(\Lambda)=2M^*M$ we see that the Parareal-based preconditioner proposed in \cite{maday2002parareal} is in fact related to the inverse Hessian of the reduced penalized objective function. If we can show that $\bar M^*\bar M$ is a positive definite matrix, we can use $Q$ as an initial approximation of the inverse Hessian in the BFGS optimization algorithm. This is however quite simple to do, as we will see in the proof of the following proposition.
\begin{proposition}
 If $\bold G_{\Delta T}$ and $\bold G_{\Delta T}^*$ are based on consistent numerical methods, $\bar M^*=\bar M^T$, and the matrix $\bar M^*\bar M$ is positive definite.
\end{proposition}
\begin{proof}
If $\bold G_{\Delta T}$ and $\bold G_{\Delta T}^*$ are based on consistent numerical methods, $\bold G_{\Delta T}(\omega)=\bold G_{\Delta T}^*(\omega)$. When inserting this into the matrices $\bar M$ and $\bar M^*$ from definition \ref{prop_def} and \ref{adjoint_prop_def}, we clearly see that $\bar M^*=\bar M^T$. For $M^*M$ to be positive definite, the following to conditions must hold:
\begin{align*}
&1.\quad x^T\bar M^*\bar Mx \geq 0 \quad \forall x\in\mathbb{R}^{N-1} \\
&2.\quad x^T\bar M^*\bar Mx =0 \iff x=0
\end{align*}
The first conditions hold due to $\bar M^*=\bar M^T$:
\begin{align*}
x^T\bar M^*\bar Mx = (\bar Mx)^T\bar Mx = ||Mx||^2 \geq 0.
\end{align*}
The second condition hold if $\bar M$ is invertible. This is true because $\bar M$ is a triangular matrix, with identity on its diagonal, and therefore has a determinant equal to 1. The determinant of a matrix being unequal to zero is equivalent with being invertible, which means that our matrix $\bar M$ is invertible, while $M^*M$ is positive definite.
\end{proof}
\subsubsection{Non-linear state equations}
As we have seen the Hessian of the non-linear problem consists of two parts. One is the linearised forward model multiplied with its adjoint, while the second part is a diagonal matrix related to the second derivative of the propagator $\bold F_{\Delta T}$, and the residuals $\lambda_i-\bold F_{\Delta T}$. The first part of $\nabla^2 \bold{\hat{J}}$ is analogue to the Hessian of the linear problem. It is symmetric positive definite, and taking its inverse corresponds to first applying the backwards model, and then the forward model. What makes the Hessian of the non-linear problematic is therefore its second term. The first issue with the diagonal matrix $D(\Lambda)$, is how to calculate $\bold F_{\Delta T}''$. Another issue is that we can not guarantee that the sum of $M(\Lambda)^TM(\Lambda)$ and $D(\Lambda)$ is a positive matrix, and the same problem would arise in a coarse approximation of $\nabla^2 \bold{\hat{J}}$. The lack of positivity is a problem since we want to use the coarse approximation as an initial inverted Hessian approximation in the BFGS-algorithm.
\\
\\
A way to get around the $D(\Lambda)$ term in the Hessian for non-linearly constrained problem, is simply to ignore it. This leaves us with the $M(\Lambda)^TM(\Lambda)$ term, which we know how to deal with. Ignoring the term depending on the second derivative and the residual is actually a known strategy for for solving non-linear least square problems. Details can be found in \cite{nocedal2006numerical}. A justification for this approach, is that at least in instances where we are close to a solution, the $\lambda_i-\bold F_{\Delta T}$ terms will be close to zero, and the $M(\Lambda)^TM(\Lambda)$ term will therefore dominate the Hessian. Ignoring the $D(\Lambda)$ term means that we can define an inverse Hessian approximation based on a coarse propagator $\bold G_{\Delta T}$ in the same way as we did for the problem with linear state equation constraints. This means that we define a matrix $\bar M(\Lambda)$:
\begin{align*}
\bar M(\Lambda) &= \left[ \begin{array}{cccc}
   \mathbbold{1} & 0 & \cdots & 0 \\  
   -\bold{G}_{\Delta T}'(\lambda_{1}) & \mathbbold{1} & 0 & \cdots \\ 
   0 &-\bold{G}_{\Delta T}'(\lambda_{2}) & \mathbbold{1}  & \cdots \\
   0 &\cdots &-\bold{G}_{\Delta T}'(\lambda_{N-1}) & \mathbbold{1}  \\
   \end{array}  \right]
\end{align*}
The term $\bar{M}(\Lambda)^{-1}\bar{M}(\Lambda)^{-*}$ can then be used in an approximation of the inverse Hessian, as detailed in section \ref{vir_sec}.
\end{document}