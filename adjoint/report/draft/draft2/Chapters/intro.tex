\chapter{Introduction}
In todays world high performance computing is an essential tool for scientist in many fileds such as engineering, comutational physics and chemistry, bioinformatics, weather forecasting and so on. Many problems that arise in these areas are so computationally costly, that they can not be solved efficiently or at all on a single processor. Instead we solve or accelerate the solution of such problems by running them on large-scale clusters of multiple processes in parallel. One of the main issues with parallel computing is that most numerical solvers are sequentially formulated, and the work of translating these algorithms into a parallel framework can often be time and effort intensive.
\\
\\
One class of large-scale problems suited for parallelization, that frequently pops up both in science and elsewhere, are time dependent partial differential equations(PDEs). The traditional approach to implementing parallel solvers for such problems is to restrict the parallel computations to operations in spacial dimension within each time step, while the time-integration is done sequentially. Letting the implementation be serial in temporal direction is the most intuitive way of parallelizing time dependent PDEs, since evolving an equation in time is a naturally sequential process. Decomposing in space on the other hand allows us to partition the problem into independent tasks. It is usually more efficient to limit the parallel computing to the spacial domain, however in cases where the number of available cores are high, introducing parallelism in temporal direction can reduce the overall solution time. We therefore want algorithms that are parallel in time.
\\
\\
There exists multiple methods for parallel in time solvers of evolution equations. The most famous and most developed of these parallel in time methods is the so called Parareal method introduced in \cite{lions2001resolution}. Other methods such as the Parareal-related multiple shooting methods\cite{bellen1989parallel,nievergelt1964parallel}, the waveform relaxation methods\cite{lelarasmee1982waveform,gander1996overlapping} and multigrid methods\cite{hackbusch1985parabolic,lubich1987multi,horton1995space} will not be touched upon in this thesis. 
\section{Project description}
Title:
Parareal in time algorithm for optimal control with PDE constraints
\\
\\
Description:
\\
\\
In optimal control with PDE constraints, we want to optimize a functional F(y(u),u) with respect to a control u and the solution to our PDE y, that depends on the control u. This problem arises in many applications such as variational data assimilation, sensitivity analysis, goal-based error estimation and more(www.dolfin-adjoint.org).
\\
\\
To minimize the functional one needs to find its gradient. Evaluating the gradient at a given control, requires solving the state equation forward in time and then the adjoint equation backwards in time. One way of parallelizing this procedure, is to partition the time interval, and then solve the equation independently on each interval. Since the equations evolve in time, the solution on each time interval depend on the previous time intervals. It is necessary to enforce this dependence in our parallel machinery. The penalty approach achieves this, by adding penalty terms to our original functional, that enforces continuity between the time intervals. We then minimize the penalized functional multiple times, increasing the penalty for non-continuity for each iteration.
\\
\\
This Msc project aims at implementing an efficient parallelization of the functional gradient evaluation, using the penalty approach. The implementation will be done in a dolfin-adjoint setting, and we want to test the algorithm on a real problem.  