\chapter{Literature review}
Prareal algorithm was introduced by Lions, Maday and Turinici in \cite{lions2001resolution} as a way to solve differential evolution equations in parallel. This is done by combining a coarse and fine scheme for discretization in time. To introduce parallelism we first decompose the time domain into subintervals and solve the equations with the fine scheme. The course scheme is used to predict the solution at the decomposed subinterval boundaries. Different applications of the algorithm exists for example on the Navier-Stokes problem\cite{fischer2005parareal} or for molecular-dynamics simulations\cite{baffico2002parallel}. The stability of the Parareal algorithm has been studied in \cite{staff2005stability} and \cite{bal2005convergence}, while results on convergence properties are given in \cite{lions2001resolution},\cite{bal2005convergence} and \cite{gander2007superlinear}. An error control mechanism for the Parareal algorithm to limit the number Parareal iterations is introduced in \cite{lepsa2010efficient}. As mentioned in \cite{gander2007superlinear} the Parareal algorithm is not the first attempt to parallelize the solution of time dependent differential equation in temporal direction, since Nievergelt already in 1964 proposed a procedure in \cite{nievergelt1964parallel} that eventually led to the so called multiple shooting methods.
\\
\\
In the paper \cite{maday2002parareal}, Maday and Turinici reinterpret the Parareal algorithm as a prconditioner for the algebraic system of equations arising when we decompose the time domain. This reinterpretation allows us to extend the parallelization of evolution equations into parallelization of optimal control with evolution equation constraints. The problem they look at is the following:
\begin{align*}
&\min_{y,u}J(y,u) = \frac{1}{2}\int_0^T||u(t)||_U^2dt + \frac{\alpha}{2}||y(T)-y^T||^2,\\
&\left\{
     \begin{array}{lr}
       	\frac{\partial y}{\partial t}+Ay = Bu\\
       	   y(0)=y_0
     \end{array}
   \right.
\end{align*}
They parallelize in time, by dividing up the time interval and enforcing continuity by adding a penalty to the functional. They then derive the adjoint equation and the gradient for the functional. The idea of Parareal, is to sequentially solve a differential equation on a coarser mesh, and then using this solution to solve the same equation in parallel. This approach is necessary to get a speed-up for the virtual problem(i.e. just solving the equation), but the coarse sequential solve also represents a bottleneck for scalability. This is addressed in \cite{rao2014adjoint}, where they try to parallelize PDEs in time in a scalable way, by only doing an initial coarse sequential solve, and thereafter doing everything in parallel.  
\\
\\
In \cite{rao2016time} the authors propose an augmented Lagrangian approach for time-parallel 4d variational data assimilation. This is basically an optimal control problem with PDE constraints. The augmented Lagrangian method is a modification of the penalty method for unconstrained optimization. Algorithms for calculating the modified functional and the gradient of the modified functional is derived, and the method is tested on a problem. The results show weak scalability for gradient and functional evaluation. 