\chapter{Literature review}
The Parareal algorithm was introduced by Lions, Maday and Turinici in \cite{lions2001resolution} as a way to solve differential evolution equations in parallel. This is done by combining a coarse and fine scheme for discretization in time. To introduce parallelism we first decompose the time domain into subintervals. The equation is first solved sequentially on the entire interval using the coarse scheme, so that we get an approximation of the solution, to be used as initial conditions for the fine solver on the decomposed subintervals. The fine solver can then be applied to each subinterval individually, and therefore also be run in parallel. This process is then repeated in an iteration, where new initial conditions for the decomposed subintervals, to be used in the parallel fine solver, are found by propagating the differences between the course and fine solutions from the last iteration, using the course scheme. 
\\
\\
Different applications of the algorithm exists for example on the Navier-Stokes problem\cite{fischer2005parareal} or for molecular-dynamics simulations\cite{baffico2002parallel}. The stability of the Parareal algorithm has been studied in \cite{staff2005stability} and \cite{bal2005convergence}, while results on convergence properties are given in \cite{lions2001resolution},\cite{bal2005convergence} and \cite{gander2007superlinear}. An error control mechanism for the Parareal algorithm to limit the number Parareal iterations is introduced in \cite{lepsa2010efficient}. One issue with Parareal, is that increasing the number of decompositions of the time domain, which we need to do if we want to increase the number of processes, will make the coarse discretization finer. This represents a possible bottleneck for speedup, which is adressed in \cite{rao2014adjoint}, where the authors try to parallelize PDEs in time in a scalable way, by only doing an initial coarse sequential solve, and thereafter doing everything in parallel. As mentioned in \cite{gander2007superlinear} the Parareal algorithm is not the first attempt to parallelize the solution of time dependent differential equation in temporal direction, since Nievergelt already in 1964 proposed a procedure in \cite{nievergelt1964parallel} that eventually led to the so called multiple shooting methods.
\\
\\
In the paper \cite{maday2002parareal}, Maday and Turinici reinterpret the Parareal algorithm as a prconditioner for the algebraic system of equations arising when we decompose the time domain. This reinterpretation allows us to extend the parallelization of evolution equations into parallelization of optimal control with evolution equation constraints. The problem they look at is the following:
\begin{align*}
&\min_{y,u}J(y,u) = \frac{1}{2}\int_0^T||u(t)||_U^2dt + \frac{\alpha}{2}||y(T)-y^T||^2,\\
&\left\{
     \begin{array}{lr}
       	\frac{\partial y}{\partial t}+Ay = Bu\\
       	   y(0)=y_0
     \end{array}
   \right.
\end{align*}
In the control problem setting the authors enforce continuity between decomposed time intervals in the state equation by adding the jumps in the solution to the objective function we want to minimize. This gives us extra variables in the objective function, and it is these variables that Parareal preconditioner effects. In \cite{rao2016time} The authors use the Parareal preconditioner approach for time-parallel 4d variational data assimilation, but enforce the state equation continuity with the augmented Lagrangian method, which is a modification of the penalty method\cite{nocedal2006numerical}. As mentioned the Parareal preconditioner presented in \cite{maday2002parareal} only affects the penalty terms. A more advanced preconditioner is derived in \cite{ulbrich2015preconditioners}, where the Parareal preconditioner is incorporated into a preconditioner for a control problem with time-dependent partiel differential equation constraints and inequality constraints.