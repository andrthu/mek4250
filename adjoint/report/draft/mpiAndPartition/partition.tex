\documentclass[11pt,a4paper]{article}
\usepackage{amsmath}
\usepackage{amssymb}
\usepackage{amsthm}
\usepackage[utf8]{inputenc}
\usepackage{graphicx}

\usepackage[utf8]{inputenc}
\usepackage[english]{babel}
\usepackage{booktabs}

\usepackage{cite}


\newtheorem{theorem}{Theorem}

\usepackage{listings}
\usepackage{color} %red, green, blue, yellow, cyan, magenta, black, white
\definecolor{mygreen}{RGB}{28,172,0} % color values Red, Green, Blue
\definecolor{mylilas}{RGB}{170,55,241}


\usepackage{graphicx}

\title{Partitioning the time interval}


\begin{document}
\maketitle
\section{The problem}
Decomposing the time interval $I=[0,T]$ into $N$ equally sized subintervals $I_i=[T_i,T_{i+1}]$, and solving the state and adjoint equations separately on each subinterval, allows our algorithm to be run in parallel. Decomposing $I$ is simple in the continuous case, however in practice we are solving these equations numerically, and in the discrete case, partitioning $I$ is more involved. To explain how we decompose $I$ in the discrete case, lets look at how to do it for the state equation. Define a differential equation $F$:
\begin{align*}
\left\{
     \begin{array}{lr}
		F(y(t),v(t))=0 \	\textit{For $t \in [0,T]$} \\
		y(0)=y_0
	\end{array}
   \right.	
\end{align*} 
We then decompose $I$, and assume that we have $N-1$ intermediate initial conditions $\{\lambda_i\}_{i=1}^{N-1}$, such that we get a solvable equation on each subinterval:
\begin{align*}
\left\{
     \begin{array}{lr}
		F^i(y_i(t),v(t))=0 \	\textit{For $t \in [T_i,T_{i+1}]$} \\
		y(T_i)=\lambda_i
	\end{array}
   \right.	
\end{align*} 
Now lets look at what happens when we discretize $I$. Lets divide $I$ into $n$ parts of length $\Delta t=\frac{T}{n}$, and set $t_k=k\Delta t$. This gives us a sequence $I_{\Delta t}=\{t_k\}_{k=0}^{n}$ as a discrete representation of the interval $I$. Using some finite difference scheme, we can transform the differential equation $F$ into a difference equation $F_{\Delta t}$:
    \begin{align*}
\left\{
     \begin{array}{lr}
		F_{\Delta t}(y^k,v(t_k))=0 \	\textit{For $k=1,...,n$} \\
		y^0=y_0
	\end{array}
   \right.	
\end{align*} 
The next step is to decompose the discrete interval $I_{\Delta t}$ into $N$ discrete subintervals. This is simply done by extracting a subsequence $\{t_{k_i}\}_{i=0}^N\subset I_{\Delta t}$ where $t_{k_0}=t_0$ and $t_{k_N}=t_n$. This results in $N$ sequences on the form $I_{\Delta t}^i=\{t_{k_i},t_{k_i+1},...,t_{k_{i+1}}\}$, and if we assume, as we did in the continuous case, that we have some intermediate initial conditions $\{\lambda_i\}_{i=1}^{N-1}$, we can solve $F_{\Delta t}$ separately on each $I_{\Delta t}^i$: 
\begin{align*}
\left\{
     \begin{array}{lr}
		F_{\Delta t}^i(y_i(t_k),v(t_k))=0 \	\textit{For $k \in \{k_i,k_i+1,...,k_{i+1} \}$} \\
		y(t_{k_i})=\lambda_i
	\end{array}
   \right.	
\end{align*} 
There is one minor issue with decomposing $I_{\Delta t}$, which I did not mention above, and that has to do with the choice of the subsequence $\{t_{k_i}\}_{i=0}^N$. In theory, one could of course freely chose any subsequence of $I_{\Delta t}$, but we generally want the difference $t_{k_i} -t_{k_{i+1}}$ to be constant for all $i$. This is however not always possible, since there is no guarantee that $n$ is divisible by $N$.
\section{Partitioning}
The general rule for partitioning a task between $N$ processes, is to distribute the work of the task as evenly as possible. The task in the above setting is solving $ F_{\Delta t}$, and the work to be distributed, is the computations required to move the solution from one time step to the next for all time steps. Since there are $n$ time steps, we should be able to say that the main task of solving $F_{\Delta t}$ can be divided into $n$ subtasks, and it is these $n$ tasks that we want to distribute between the $N$ processes. Now deciding how many subtasks each process should get is quite simple. Start with defining the numbers:
\begin{align*}
q &= \lfloor \frac{n}{N}\rfloor \\
r &= N \mod n
\end{align*} 
Then we give each process $q$ tasks, and then add one task to $r$ processes. To which processes you give the extra task does not really matter, but the most straightforward way of doing it is just to give the first $r$ processes the extra task. I however chose to look at the distributing problem slightly different, by instead of trying to distribute the $n$ tasks, looking at how to evenly divide the $n+1$ points $\{t_k\}_{k=0}^{n}$ among the $N$ processes. Again lets define $q$ and $r$ as:
\begin{align*}
q &= \lfloor \frac{n+1}{N}\rfloor \\
r &= N \mod n+1
\end{align*}
Every process now gets $q$ points, but due to overlap every process gets an extra point excluding the first process. Then the remaining $r$ points are given to the first $r$ processes. This allows me to define the sequence $\{k_{i}\}_{i=0}^N$ recursively as follows:
\begin{align*}
k_{i+1} = k_i + q+\delta_{i\neq0}+\delta_{i<r}
\end{align*}
Here the $\delta$s are conditional functions defined as:
\begin{align*}
\delta_{S}=\left\{
     \begin{array}{lr}
		1 \ S=\text{True} \\
		0 \ S=\text{False}
	\end{array}
   \right.	
\end{align*} 
We now have a way of decomposing the discrete time interval, and therefore also a way of partitioning the finite difference solver of the state equation in temporal direction. However both when we want try to find the gradient in a penalized optimal control problem, or when we just want to parallelize solving a differential equation using the parareal scheme, some communication between the processes is required. 
\section{Communication}
\end{document}