\documentclass[11pt,a4paper]{article}
\usepackage{amsmath}
\usepackage{amssymb}
\usepackage{amsthm}
\usepackage[utf8]{inputenc}
\usepackage{graphicx}

\usepackage[utf8]{inputenc}
\usepackage[english]{babel}
\usepackage{booktabs}

\usepackage{cite}


\newtheorem{theorem}{Theorem}

\usepackage{listings}
\usepackage{color} %red, green, blue, yellow, cyan, magenta, black, white
\definecolor{mygreen}{RGB}{28,172,0} % color values Red, Green, Blue
\definecolor{mylilas}{RGB}{170,55,241}


\usepackage{graphicx}

\title{Partitioning the time interval}


\begin{document}
\maketitle
\section{The problem}
Decomposing the time interval $I=[0,T]$ into $N$ equally sized subintervals $I_i=[T_i,T_{i+1}]$, and solving the state and adjoint equations separately on each subinterval, allows our algorithm to be run in parallel. Decomposing $I$ is simple in the continuous case, however in practice we are solving these equations numerically, and in the discrete case, partitioning $I$ gets more involved. To explain how we decompose $I$ in the discrete case, lets look at how do it for the state equation. Define a differential equation $F$:
\begin{align*}
\left\{
     \begin{array}{lr}
		F(y(t),v(t))=0 \	\textit{For $t \in [0,T]$} \\
		y(0)=y_0
	\end{array}
   \right.	
\end{align*} 
We then decompose $I$, and assume that we have $N-1$ intermediate initial conditions $\{\lambda_i\}_{i=1}^{N-1}$, such that we get a solvable equation on each subinterval:
\begin{align*}
\left\{
     \begin{array}{lr}
		F^i(y_i(t),v(t))=0 \	\textit{For $t \in [T_i,T_{i+1}]$} \\
		y(T_i)=\lambda_i
	\end{array}
   \right.	
\end{align*} 
Now lets look at what happens when we discretize $I$. Lets divide $I$ into $n$ parts of length $\Delta t=\frac{T}{n}$, and set $t_k=k\Delta t$. This gives us a sequence $I_{\Delta t}=\{t_k\}_{k=0}^{n}$ as a discrete representation of the interval $I$. Using some finite difference scheme, we can transform the differential equation $F$ into a difference equation $F_{\Delta t}$:
    \begin{align*}
\left\{
     \begin{array}{lr}
		F_{\Delta t}(y^k,v(t_k))=0 \	\textit{For $k=1,...,n$} \\
		y^0=y_0
	\end{array}
   \right.	
\end{align*} 
\end{document}