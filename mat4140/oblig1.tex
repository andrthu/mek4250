\documentclass[11pt,a4paper]{report}
\usepackage{amsmath}
\usepackage{amssymb}

\begin{document}
\begin{center}

\LARGE INF5620 - Mandatory assignment 1
\\
Andreas Thune
\\
\LARGE
16.02.2016

\end{center}
\textbf{4.1.1}
\\
Want to interpolate the the pairs $\{(x_i,f_i \}_{i=1}^n$, with a polynomial $p \in P_n$, by using the basis $\{ x^{i-1}\}_{i=1}^n$. Want $p(x_i)=f_i$ $\forall$ i. Need $\{ c_i\}_{i=1}^n$ s. t. $p(x) = \sum_{i=1}^n c_ix^{i-1} $ has the described property. this means, that we need to solve the system $$\sum_{j=1}^n c_jx_i^{j-1} = f_i \ \forall \ i=1...n $$ This gives us the matrix

$$ 
V^T = 
 \begin{pmatrix}
  1 & x_1 & x_1^2 & \cdots & x_1^{n-1} \\
  1 & x_2 & x_2^2 &\cdots & x_2^{n-1} \\
  \vdots  & \vdots &\vdots  & \ddots & \vdots  \\
  1 & x_n & x_n^2 &\cdots & x_n^{n-1} 
 \end{pmatrix}
$$
And the system $$V^Tc=f$$ when $c=(c_1,...,c_n)^T$ and $f=(f_1,...,f_n)^T$. Notice that $V^T$ is the traverse of the Vandermonde matrix. This means that $V_{i,j} = V_{j,i}^T = x_j^{i-1}$
\\
\\
\textbf{4.1.3}
\\
The definition of o triangular family of polynomials is a set of polynomials with the following form $$Q=\{ \sum_{i=1}^js_{i,j}x^{i-1}\}_{j=1}^n, \ s_{j,j} \neq 0$$ $Q$ being a basis for $P_n$ is equivalent to $\{S_i\}_{i=1}^n$ being a basis for $\mathbb{R}$, when $S_j = (s_{1,j},s_{2,j},...,s_{j,j},0,...,0)^T$. This is again equivalent to the matrix $$ 
A = 
 \begin{pmatrix}
  s_{1,1} & s_{1,2} & s_{1,3} & \cdots & s_{1,n} \\
  0 & s_{2,2} & s_{2,3} &\cdots & s_{2,n} \\
  0 & 0 & s_{3,3} &\cdots & s_{3,n} \\
  \vdots  & \vdots &\vdots  & \ddots & \vdots  \\
  0 & 0 & 0 &\cdots & s_{n,n} 
 \end{pmatrix}
$$ having determinant unequal to zero. This is true, since $\ s_{i,i}\neq 0$ and the determinant of a triangle matrix is the product of the diagonal, i. e. $$det(A)=\prod_{i=1}^n s_{i,i}\neq 0$$ 
\end{document}