\documentclass[11pt,a4paper]{report}
\usepackage{amsmath}
\usepackage{amssymb}

\usepackage{graphicx}

\usepackage{listings}
\usepackage{color} %red, green, blue, yellow, cyan, magenta, black, white
\definecolor{mygreen}{RGB}{28,172,0} % color values Red, Green, Blue
\definecolor{mylilas}{RGB}{170,55,241}


\usepackage{graphicx}


\begin{document}
\begin{center}

\LARGE Mat-Inf4140 - Mandatory assignment 1
\\
Andreas Thune
\\
\LARGE
16.02.2016

\end{center}
\textbf{4.1.1}
\\
Want to interpolate the the pairs $\{(x_i,f_i \}_{i=1}^n$, with a polynomial $p \in P_n$, by using the basis $\{ x^{i-1}\}_{i=1}^n$. Want $p(x_i)=f_i$ $\forall$ i. Need $\{ c_i\}_{i=1}^n$ s. t. $p(x) = \sum_{i=1}^n c_ix^{i-1} $ has the described property. this means, that we need to solve the system $$\sum_{j=1}^n c_jx_i^{j-1} = f_i \ \forall \ i=1...n $$ This gives us the matrix

$$ 
V^T = 
 \begin{pmatrix}
  1 & x_1 & x_1^2 & \cdots & x_1^{n-1} \\
  1 & x_2 & x_2^2 &\cdots & x_2^{n-1} \\
  \vdots  & \vdots &\vdots  & \ddots & \vdots  \\
  1 & x_n & x_n^2 &\cdots & x_n^{n-1} 
 \end{pmatrix}
$$
And the system $$V^Tc=f$$ when $c=(c_1,...,c_n)^T$ and $f=(f_1,...,f_n)^T$. Notice that $V^T$ is the traverse of the Vandermonde matrix. This means that $V_{i,j} = V_{j,i}^T = x_j^{i-1}$
\\
\\
\textbf{4.1.3}
\\
The definition of o triangular family of polynomials is a set of polynomials with the following form $$Q=\{ \sum_{i=1}^js_{i,j}x^{i-1}\}_{j=1}^n, \ s_{j,j} \neq 0$$ $Q$ being a basis for $P_n$ is equivalent to $\{S_i\}_{i=1}^n$ being a basis for $\mathbb{R}$, when $S_j = (s_{1,j},s_{2,j},...,s_{j,j},0,...,0)^T$. This is again equivalent to the matrix $$ 
A = 
 \begin{pmatrix}
  s_{1,1} & s_{1,2} & s_{1,3} & \cdots & s_{1,n} \\
  0 & s_{2,2} & s_{2,3} &\cdots & s_{2,n} \\
  0 & 0 & s_{3,3} &\cdots & s_{3,n} \\
  \vdots  & \vdots &\vdots  & \ddots & \vdots  \\
  0 & 0 & 0 &\cdots & s_{n,n} 
 \end{pmatrix}
$$ having determinant unequal to zero. This is true, since $\ s_{i,i}\neq 0$ and the determinant of a triangle matrix is the product of the diagonal, i. e. $$det(A)=\prod_{i=1}^n s_{i,i}\neq 0$$ 
\\
\\
\textbf{4.1.5}
\\
Using overdetermination, means that we are using more interpolation points then the power of the polynomial we want to interpolate with, i.e. we want $p \in P_n$ such that $p(x_i)=f_i$ for $i=1,...,m$, where $m>n$. In general this is not possible to achieve, but we can find a polynomial $p$ such that $p(x_i)$ is as close as possible to $f_i$ in a least square sense. This is the overdetermined system. In the book it says that positive features of solutions of such systems, is that it reduces the effect of irregular function values from f, and that it gives the polynomial a smoother behavior between gridpoints.
\\
\\   
\textbf{Computer exercise 1.1}
\\
a) All the code and images will be pasted below. For the the absolute error in the first part of a), you see that the error is almost the same for the two types of points. For the second part I choose to add a random number between 0 and $0.01$ to the function value. I then see that when I use a large amount of points to interpolate (I used 20 in the code), the error between the polynomial I get with perturbations and the polynomial I get without perturbations, is in the same magnitude as $0.01$ for chebychev points, but for equidistant points, the error blows up close to the boundary. This you can see in the image below. This can mean that we have stability in sup norm for chebychev points, but not for equidistant points.
\\   
\\
b) Let f take random values between 0 and 1. For chebychev points the polynomial lies approximatly between 0 and 1 between the grid points. However for equidistant points, the polynomial gets much bigger than this between the points especially if we use many points. 



\lstset{language=Matlab,%
    %basicstyle=\color{red},
    breaklines=true,%
    morekeywords={matlab2tikz},
    keywordstyle=\color{blue},%
    morekeywords=[2]{1}, keywordstyle=[2]{\color{black}},
    identifierstyle=\color{black},%
    stringstyle=\color{mylilas},
    commentstyle=\color{mygreen},%
    showstringspaces=false,%without this there will be a symbol in the places where there is a space
    numbers=left,%
    numberstyle={\tiny \color{black}},% size of the numbers
    numbersep=9pt, % this defines how far the numbers are from the text
    emph=[1]{for,end,break},emphstyle=[1]\color{red}, %some words to emphasise
    %emph=[2]{word1,word2}, emphstyle=[2]{style},    
}


\section*{Code for finding error in 4.1.1 a)}
\lstinputlisting{o1.m}
\section*{Small pertubation test}
\lstinputlisting{per.m}
\section*{Exercise b}
\lstinputlisting{ob.m}



\begin{figure}
  \includegraphics[width=\linewidth]{opp1.png}
  \caption{Error between function and interpolating polynomial}
  \label{Fig 1}
\end{figure}

\begin{figure}
  \includegraphics[width=\linewidth]{per.png}
  \caption{Error between polynomial with and without perturbations}
  \label{Fig 2}
\end{figure}

\begin{figure}
  \includegraphics[width=\linewidth]{opp2.png}
  \caption{Plot of interpolation polynomial of random function in $[0,1]$ }
  \label{Fig 3}
\end{figure}

\end{document}